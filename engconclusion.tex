\chapter{Conclusions}
\label{cap:conclu}

\section{Algorithm for Virtual Patient Positioning}
\label{conclu:posing}

Nowadays, virtual reality is proving its potential in the medical training  field.  
%En la actualidad, la \acs{RV} está demostrando su potencial como herramienta de entrenamiento en el campo de la medicina. %En este contexto, el algoritmo propuesto ofrece la posibilidad de adaptar la postura de cualquier paciente virtual a la posición requerida por el procedimiento médico que se va a simular.
In this thesis, we have proposed a semiautomatic animation pipeline which allows adapting a virtual patient to the required pose in a given medical procedure dealing with internal tissues.  
%En esta tesis se ha presentado un cauce de animación esqueletal semiautomático que permite transformar cualquier paciente virtual con estructuras internas a la posición requerida en el entrenamiento de un determinado procedimiento médico.

We follow a geometric-based approach. Therefore, it provides a heuristic solution and does not represent the correct physical behaviour of internal tissues. Although musculoskeletal  simulation  has  significantly  advanced in  the  computer  graphics field, those techniques usually require an accurate description of the patient tissues, but sometimes they are not available or they are incomplete.

%Esta técnica sigue un enfoque geométrico y, por tanto, proporciona una solución heurística que no refleja  el comportamiento físico de los tejidos. Aunque recientemente las simulaciones músculo-esqueletales han avanzado mucho en el mundo de los gráficos por computador, estas técnicas requieren de una descripción, tanto mecánica como geométrica, detallada y completa de los tejidos a animar, los cuales en ocasiones no se encuentran disponible. 
%quitar salto de párrafo
%El algoritmo propuesto ofrece dos ventajas claves. Primero, su flexibilidad a la hora de transformar un paciente virtual junto con sus tejidos internos aunque estos se encuentren incompletos o falte su descripción mecánica. Segundo, su selección de poses se ejecuta con tasas de refresco interactivas. 

One advantage of the proposed algorithm is its flexibility. Only skin and bones are required. That kind of tissues are usually registered by most of the medical imaging techniques. Our approach computes a displacement field from the bones' movement. This field is used to transform the internal tissues. Our algorithm works with incomplete anatomical models and does not need a mechanical description of the tissue behaviours. The technique is flexible enough to deal with \acs{B-rep}s and volumetric models (see sec. \ref{posing:animvol}). Thus, it extends the scope of the proposed method.

%Una de las ventajas del algoritmo propuesto es que se diseñó con el objetivo de ser lo más flexible posible, minimizando los requisitos de los datos de entrada. La clave de está flexibilidad radica en que se calcula un campo de deformaciones continuo dentro del modelo anatómico. Este campo permite adaptar cualquier tejido a la pose deseada, de forma independiente al resto de tejidos, pudiendo así trabajar con modelos incompletos. En este sentido, la única limitación que se impone es que tanto la piel como el tejido óseo del paciente virtual deben de estar segmentados. Esta restricción no debería suponer un gran problema, dado que estos tejidos son visibles en la mayor parte de las técnicas de imagen médica. Además, tal y como se muestra en la sección \ref{posing:animvol}, este campo de deformaciones puede aplicarse tanto a \acs{B-rep}s como a datos volumétricos. De esta manera, se ha extendiendo el alcance de la técnica propuesta.

Using the displacement field yield some advantages. It is mandatory to avoid self-collisions and collisions among other tissues, in order to ensure proper operation of the physics simulation on \acs{RASim}, and to enable to produce more realistic images in the ultrasound (US) module or in \emph{gVirtualXRay} library. This means collision and self-collisions will appear if input data are not collision-free or tissues meshes have a poor resolution. 

%La utilización del campo de deformaciones aporta beneficios adicionales. Evitar colisiones y autocolisiones en los tejido transformados es de vital importancia de cara a garantizar el buen funcionamiento del simulador físico y conseguir imágenes más realistas tanto por el módulo de simulación de \acs{US} de \acs{RASimAs} como por la librería \emph{gVirtualXRay} en el simulador de radiología diagnóstica. De esta forma, solo podrán aparecer colisiones y/o autocolisiones si: (i) se utilizan mallas para representar los tejidos con una resolución inadecuada o (ii) si existen colisiones y/o autocolisiones en el modelo virtual de partida. 
Due to the medical imaging techniques are not perfect, tissues usually show collisions. Collisions with bones or skin are specially harmful and they lead to unrealistic results. Our algorithm deals with this kind of artefacts, and it is more robust than physically-based simulation techniques.

%Debido a que la extracción de los tejidos desde imágenes médicas no es perfecta, pueden aparecer las auto colisiones y colisiones entre tejidos. La técnica propuesta no solventa estas colisiones, pero es robusta en su tratamiento. Aun así, las colisiones con el tejido óseo y muy especialmente con la piel, pueden provocar una importante perdida de realismo. Cabe destacar, que, a pesar de esta circunstancia, la técnica propuesta es mucho más robusta, en este escenario, que los métodos basados en simulación física. %En la sección \ref{posing:method}, se proponen técnicas para mitigar estos problemas. 




% El algoritmo se ha diseñado para ser robusto y poder adaptar la posición de cualquier modelo anatómico. Además, se permite utilizar tanto mallas superficiales como modelos volumétricos. Únicamente, se considera necesario el tejido óseo y la piel del modelo anatómico debido a que la animación se dirige por los huesos que contenga el paciente virtual.
%quitar salto de párrafo
%Las primeras etapas de este cauce se realizan en un preproceso automático que generan la información necesaria para que, a continuación, el usuario pueda seleccionar la postura del paciente virtual interactivamente.

Another advantage of the proposed method is that achieves interactive frame rates. Therefore, this technique can be used to animate the internal anatomy of virtual characters in any interactive applications, such as medical simulator, video games, etc. All the computationally expensive stages are grouped in an automatic pre-process stage, while animation and skinning stages run at interactive rates.
%La otra ventaja que presenta el algoritmo propuesto es su eficiencia computacional que permiten utilizarlo en aplicaciones interactivas. Se ha diseñado el cauce de tal forma que las etapas que no pueden ejecutarse en tiempo real se realicen de forma automática en preproceso, permitiendo así, que el usuario pueda seleccionar la pose del paciente virtual en interactivamente.
%Yo quitaría esta última frase. No se muy bien que aporta. 
% \del{Cabe destacar las innovaciones incluidas en cada etapa del cauce de animación esqueletal.}
Those automatic stages are very flexible and they could be accommodated to different problems. % except rigging phase is a specific solution for the ITGVPH. Still, the proposed pipeline could be easily extended if it exist a direct correspondence between patient bones and virtual skeleton.
%La etapas del cauce propuesto son suficientemente flexibles como para poder adaptarse una gran diversidad de problemas, siendo la fase de \emph{rigging} la que más limitaciones impone debido a que es una solución especifica diseñada para integrase en el generador de pacientes virtuales de \acs{RASim}. Aun así, este método puede ser fácilmente extendido, siempre y cuando se pueda establecer una correspondencia directa entre el tejido óseo y el esqueleto virtual.
%ampliado, aunque el único requerimiento es haya una una correspondencia directa entre el hueso virtual y los tejidos óseos  presentes en el modelo anatómico. Si no fuera así, se podría generar deformaciones no realistas.

%La etapa de volumetrización consiste en discretizar el interior del modelo para obtener una malla de tetraedros que será la base del campo de deformaciones anteriormente citado.

In the weighting stage, we have adapted the formulation \cite{Baran:2007}  to volumetric models. \emph{Baran and Popovi\'{c}} use Laplace diffusion equation to calculate the bone's influence smoothly along the superficial mesh, and we have changed their approach to deal with tetrahedral meshes. Additionally, our technique uses the bone's tissues instead of virtual skeleton  to control the animation process.

%En cuanto a la fase de pesado, se ha basado en la propuesta de \emph{Baran y Popovi\'{c}}. En \cite{Baran:2007} se proponía utilizar la ecuación de difusión del calor para calcular los pesos automáticamente. En el cauce propuesto, esta ecuación ha sido adaptada para ser utilizada en mallas volumétricas en comparación con las mallas superficiales usadas originalmente. Además, nuestra técnica utiliza la geometría del tejido óseo en lugar de la geometría del esqueleto virtual.

%Hay que señalar los resultados obtenidos en la fase de \emph{skinning}. 

The skinning phase of our pipeline is very fast, particularly considering the complexity of the used patient models. In comparison with the fastest physically-based models \cite{Bender:2014}, our algorithm offers interactive rates with meshes of one order of magnitude higher. Among the available \emph{skinning} techniques, we have chosen \acs{COR} \cite{le2016real} because it solves most of the artefacts introduced by \acs{LBS} \cite{thalmann88} or \acs{DQS} \cite{Kavan2008}. We have adapted the original \acs{COR} method to deal with tetrahedral meshes. This method requires to add a new stage to the pre-process module.% One more stage have been added to the pre-process.

%La etapa de \emph{skinning} consigue tasas de refresco interactivas altas, incluso en modelos con alto grado de detalle. Frente a los métodos basados en modelos físicos más eficientes \cite{Bender:2014}, el algoritmo propuesto ofrece un mejor rendimiento utilizando mallas de un orden de magnitud superior. Entre las técnicas de \emph{skinning} disponibles, se ha elegido \acs{COR} \cite{le2016real} debido a que soluciona los defectos más comunes de las técnicas \acs{LBS}\cite{thalmann88} y \acs{DQS}\cite{Kavan2008}. Esta técnica ha sido adaptada para tratar las mallas de tetraedros, por lo cual se ha incorporado una etapa de calculo de \acl{COR} al cauce. Aun así, el cauce permite incorporar cualquier técnica de puede utilizarse cualquier técnica de \emph{skinning} basada en pesos.
%Esta complejidad es en ocasiones de un orden superior en comparación con los métodos de animación basados en físicas. 
%Por una parte, modelos precisos raramente consiguen tiempos interactivos, y por otra parte, recientemente aparecidos modelos basados en dinámicas de puntos usan modelos mucho más complejos para ser interactivos  \cite{abu2015position}.
%Quitar salto de parrafo
%\todo{Comenta algo de la adapatación de CoR, porque se escogio esta técnica y comenta que puede utilizarse cualquier técnica de skinning basada en pesos.}

Additionally, in order to refine the solutions
provided by our geometric approach, we have designed an optimisation phase to achieve more appealing results. We have adapted a  physically-based method to guarantee volume conservation. We use a co-rotational FEM formulation to solve the steady-state problem for a linear, isotropic and homogeneous materials.

%Siendo conscientes de las limitaciones de estos métodos %de \emph{skinning}, 
%se incorporó una fase de optimización que pretendía mejorar los resultados obtenidos. Esta etapa consiste en utilizar un modelo matemático basado en mecánica de medios continuos, con el único objetivo de garantizar la conservación del volumen.
%Sin poder utilizar propiedades mecánicas  del modelo, el objetivo era garantizar la conservación del volumen en la malla volumétrica. 
%Esta etapa podría ser modificada para tomar en cuenta las propiedades mecánicas de los tejidos en el momento de generar la malla volumétrica y así poder mejorar el realismo del método.

It is important to highlight that our geometric approach
provides a heuristic solution which is not suitable for surgical planning. Alternatively, students do not need a specific patient model but a set of plausible virtual patients. This could be used for training and educational purposes. The proposed algorithm could be integrated into novel simulators which will improve student training instead of the traditional and static learning ways such as books.
%Es importante también tener en cuenta que una solución heurística no produce resultados físicamente correctos. Es por esta razón que esta técnica no es adecuada para planificación quirúrgica. Sin embargo, los entrenadores médicos no necesitan específicamente modelos de pacientes reales sino una muestra variada de modelos anatómicos que permitan entrenar casos plausibles. Estos modelos se podrán utilizar en herramientas educativas y de aprendizaje para formar a estudiantes de una manera innovadora que no proporcionan los materiales clásicos como los libros.
%Sin salto de parrafo

To demonstrate the algorithm capabilities, two use cases have been proposed to prove the feasibility of the inclusion of our method in virtual reality solutions.
%Para demostrar las distintas capacidades del algoritmo presentado, se han presentado dos casos de uso que permiten comprobar la viabilidad la utilidad del algoritmo propuesto en soluciones de \acs{RV} reales. 
%
On one hand, the proposed technique has been integrated into the \acs{ITGVPH} where it allows generating a database of VPH. %  medical professional to animate and transform interactively the virtual patient generated.
%Por una parte, se ha incorporado el algoritmo en una \emph{suite} de aplicaciones que permite crear una base de datos de pacientes virtuales.
%incorporados en el simulador \acs{RASim}.
\acs{ITGVPH} creates virtual patients based on patient-specific data in order to use them in \acs{RASim}.
%\acs{ITGVPH} crea modelos que representen distintas  variaciones anatómicas registrando un modelo de paciente virtual e imágenes médicas provenientes de pacienes reales. 
%procedentes del registro entre un modelo comercial e imágenes médicas. 
%El algoritmo permite 
%modificar % no se modifica "a", se adapta "a"
%adaptar la postura de este modelo a la requerida por el simulador.
%
On the other hand, our algorithm has been incorporated into an X-ray projectional radiograph simulator. In this application, the user adapts the patient pose as a part of the medical procedure. Real-time interaction is required. %This application needs user can animate a virtual patient in real time as part of the projectional radiograph procedure meanwhile he is watching the X-ray interactively.
%Por otra parte, el algoritmo se ha incorporado en simulador de radiología diagnóstica. Esta aplicación requiere que el usuario modifique la pose del modelo virtual de forma interactiva como parte del procedimiento simulado a la vez que permite observar la radiografía en tiempo real. %\del{que necesita deformaciones en tiempo real y permitan al usuario interaccionar con los modelos virtuales.}


%Por último, aunque los campos de uso están orientados en el área de la medicina, esta técnica no está limitada a este ámbito en concreto, sino que podría ser empleada para animar modelos anatómicos con estructuras internas en otras áreas como los videojuegos, la industria del cine, etcétera.
 

\section{RASimAs}


\subsection{ITGVPH}
\label{conclu:herramienta}

% \todo{indica que has creado un la aplicación de selección de poses que usa tu algoritmo. Indica que has trabajado en en la integración del entorno.}

The goal of the \acs{ITGVPH} is to generate a \acs{VPH} database which would be used by \acs{RASim}. \acs{ITGVPH} is able to generate new \acs{VPH} from existing virtual models and patient-specific data to provide anatomic variation. The \acs{TPTVPH} has been integrated into the toolkit and it allows the user to interactively adapt the VPH through a UI. Users can choose a pose required by the procedure.

%El objetivo de la \emph{suite} \acs{ITGVPH} es generar una base de datos de \acs{VPH} usados por el simulador \acs{RASim} con la finalidad de entrenar con una variabilidad anatómica extensa.
%\acs{ITGVPH} es capaz de generar nuevos \acs{VPH} a partir de modelos anatómicos virtuales existentes y de imágenes médicas que proporcionen variabilidad anatómica.  
%Se ha trabajo en el desarrollo y la integración de la herramienta \acs{TPTVPH} en la \emph{suite}, la cual proporciona una \acs{IU} al algoritmo propuesto. En esta, se permite adaptar interactivamente la postura del \acs{VPH} a la posición requerida por el procedimiento que se va a entrenar.
 
% Esta herramienta proporcionaría la ventaja de permitir a los estudiantes practicar con multitud de escenarios de forma segura. La creación una base de datos con pacientes virtuales puede ser clave en la mejora de los simuladores de \acs{RV} para el campo de la medicina. 

%Esta herramienta está compuesta por tres módulos desarrollados por separado. Uno de ellos es el algoritmo de posicionamiento de pacientes virtuales que se ha presentado en esta tesis. Estos módulos han sido creados de manera independiente.
 %Por último, se generarán tanto los modelos superficiales como volumétricos con los parámetros físicos que se quieran simular. Estos modelos describen al \acs{VPH} anatómicamente y mecánicamente para que puedan ser utilizados en el proyecto europeo. Además, esta herramienta permite la ejecución en serie o de manera independiente los módulos. Debido a que en ocasiones pueda ser posible utilizar ciertas etapas sin necesidad de ejecutar el cauce completamente.


The main goal is to reduce the user interaction to the minimum. % of a non-technical user leading to a fully automatic procedure. 
All the steps are automatic except the selection pose stage, where the physician supervises the proper animation of the VPH tissues and its usefulness. \acs{ITGVPH} will sequentially run until the user must be required to select the pose of the VPH in the \acs{TPTVPH} tool. Users could generate as many poses as they want through a user-friendly application with the aim of minimising the spent time. Even, they can choose to run the optimisation phase in order to refine the result. When they finish, \acs{ITGVPH} continues executing the remaining phases to generate the VPH which will be used by \acs{RASim}

%Debido a que se requiere limitar la interacción del usuario con un perfil no técnico se ha automatizado todas las etapas de \acs{ITGVPH} excepto la selección de poses. Al basarse en un método geométrico, es responsabilidad del profesional clínico seleccionar transformaciones útiles para el entrenamiento.   Según los parámetros de entrada, la \emph{suite} ejecuta los distintos módulos secuencialmente, requiriendo la interacción del usuario para seleccionar la postura del paciente virtual en la herramienta \acs{TPTVPH}. El supervisor clínico puede generar tantas posturas como sea necesario a través de la \acs{IU} que muestra la escena 3D donde se puede observar el paciente virtual. Incluso, se puede ejecutar la fase de optimización para refinar el resultado. Una vez adaptado el \acs{VPH}, la \emph{suite} continua con la ejecución de las tareas restantes para generar los modelos necesario que se utilizarán en el simulador \acs{RASim}.


%En esta, se puede configurar las etapas de manera independiente. El usuario es capaz de seleccionar el modelo anatómico de referencia (p. ej Zygote), las imágenes médicas en formato \acs{DICOM} (p. ej imágenes de \acs{IRM}), la postura necesaria para el procedimiento y, por último, especificar parámetros biomecánicos. 
%Las posturas de los \acs{VPH} son definidas por un En esta interfaz, el usuario podrá interaccionar con las articulaciones disponibles para generar unas posiciones físicamente correctas.

\subsection{RASim}
\label{conclu:rasim}

% \todo{3 Apunta en las conclusiones una discusión sobre los problemas de RAsimas que se alcanzo que no y porque.}

\acs{RASim} simulator provides a complete training environment to practise the  \acs{US} guided \acs{RA}. A “T” shape base platform is proposed for the RASim workbench. The positioning of the haptic device and the mannequin are relative to each other and can be moved accordingly for left and right handed operators. Two monitors show the current rendered virtual scene and the \acs{Courseware} view with the ultrasound images. Users would perform the scout scan, needle guidance and injection phases of the \acs{RA} procedure. One of this thesis' contribution is the design and development of the \acs{Courseware} module. This manages and communicates all the \acs{RASim} components, providing a self-training and evaluation platform. \acs{Courseware} allows registering metrics in order to give formative and summative feedback before, during and after the simulation.


%El simulador \acs{RASim} proporciona un entorno de entrenamiento completo y realista para el entrenamiento del procedimiento de \acs{RA} guiado por \acs{US}.  Este simulador cuenta con una mesa de trabajo donde el usuario puede realizar el procedimiento con instrumentos que imitan un entrono real.
%En los monitores podrá supervisar la interacción de estos y el paciente, mientras puede observar la imagen simulada de \acs{US}. El estudiante podrá realizar las fases de exploración, guiado de la aguja e inyección del procedimiento de \acs{RA}. Una contribución de esta tesis ha sido el diseño y desarrollo del módulo  \acs{Courseware}, el cual gestiona todos los componentes del simulador. Este módulo es el encargado de proveer una plataforma de entrenamiento y autoevaluación, guiando al estudiante en su proceso de aprendizaje. Este módulo también monitoriza métricas de rendimiento con el objetivo de ofrecer retroalimentación formativa y sumativa. 

%\todo{
%Aaron tienes que ser mas estructurado. Te pongo las ideas que tienes que contar y tu las das formas:
%1. Dos objetivos importantes:
%1.1 Proporcionar retroalimentacion haptica. Es importante porque ayuda a localizar la aguja
%.2 Mantener el precio del simulador bajo. 
%2. Se escogió el dispositivo phantom (deberías haberlo contarlo en Rasimas cuando) proporcionar un dispositivo haptico de bajo precio. Cumple los dos objetivos
% 3. Se encontraron problemas en los nuevos dispositovos (pon un anexo con la carta de  Geomagic indicando el error). Yo daria detalles explicando los problemas de precisión.
% 4. Esto problema impidieron la validación del prototipo y por lo tanto de la hipótesis. 
%5. La URJC y el doctorando apataron apataron el prototipo para utilizar un dispositivo flock of birds. 
%Este dispositivo no proporciona retroalimentacion haptica pero es muy precsiso. 
%6. Los socios medicos validaron este prototipo a pesar de no tener retroalimentación.
%7. Lamentablemente se consumieron los plazos y puedo concluierse el resto de evaluaciones que estaban probramadas quedandonos soloamanete con el visto bueno de los socioes medicos. }


Before reaching the clinical trial milestone, every module
of the simulator have been beta tested upon its integration to ensure the evaluation process. This has allowed an iterative development approach securing delivery of a validated and reliable final prototype. One goal of the \acs{RASim} is to simulate the haptic feeling of the needle. This functionality was expected by the medical partners. The consortium selected a low-cost haptic device to this porpose. % to improve users' efficiency. Touch device were selected as low-cost haptic device. %Uno de los objetivos del simulador \acs{RASim} era poder simular la sensación háptica de la aguja. Esta funcionalidad era muy esperada por el socios médicos, porque esto implicaría ayudar a los estudiantes a ser más eficientes en el procedimiento. Se seleccionó el dispositivo háptico \emph{Touch} como solución de bajo coste para la devolución de fuerzas al usuario.  
Unfortunately, those devices showed manufacturing defects (see sec. \ref{result:rasim}), which lead to a mismatch between virtual scene and the real devices. For that reason, doctors did not allow the use of the simulator by students in order to avoid bad practices.

%Desafortunadamente, este dispositivo mostró defectos de fábrica (ver sec. \ref{result:rasim}), que se traducía en una falta de correspondencia entre la posición de los instrumentos virtuales y los dispositivos. Esto impidió que los médicos permitieran su utilización por parte de los estudiantes.  
%
From \acs{URJC}, we proposed a solution for replacing the haptic device by a magnetic tracker. Although they do not provide haptic feedback, they are very accurate. In this way, students could train their propioceptive skills on the simulator. Medical partners approve this alternative in the absence of a suitable haptic device.
%Desde la \acs{URJC} se trabajó para proponer una solución utilizando \acs{tracker}s magnéticos. Aunque estos dispositivo no proporciona retroalimentación háptica pero es muy preciso y, de esta manera, los estudiantes podrían practicar sus habilidades propioceptivas. Los socios médicos dieron el visto bueno a esta alternativa ante la falta de un dispositivo háptico. 

All module were tested by the medical committee but \acs{RASimAs}' schedule was over without the possibility to perform the clinical trials.

%Lamentablemente, aunque se pudieron concluir las evaluaciones programadas de los distintos componentes del simulador habiendo recibiendo el visto bueno de los socios médicos, se consumieron los plazos del proyecto europeo que finalizó sin poder realizar el ensayo clínico.%, con la cual permitiría asegurar que el simulador \acs{RASim} proporcionaría un entorno de entrenamiento completo para estudiantes o profesionales de la anestesia. 

%a través de los dispositivos utilizados en el simulador permiten hacerse una idea de que se van a encontrar en un procedimiento real. 

Finally, it is worth to mention that the European Commission evaluated the \acs{RASimAs} project as \emph{Acceptable progress}. Clinical trials were not conducted because of unexpected problems. Even so, they valued positively the \acs{RAAs} system and the final condition of the \acs{RASim} prototype. European Commission encourages to finish any unfinished task in order to benefit anaesthesia students.
%Es importante comentar el resultado final de la evaluación de la comisión europea sobre el proyecto \acs{RASimAs}. El comité evaluador calificó el proyecto con un \emph{progreso aceptable} debido a las dificultades para finalizar la evaluación clínica. Los problemas inesperados fueron los determinantes para que esta evaluación no pudiera hacerse en el tiempo esperado. Aun así, se valoró muy positivamente el módulo de \acs{RAAs} y el estado del prototipo final de \acs{RASim}. El propio comité animó a seguir con los trabajos necesarios que hagan falta para la finalización del \acs{RASim} y que pueda ser utilizado en entornos reales para el beneficio de los estudiantes de anestesiología.


\section{X-ray projectional radiography simulator }
\label{conclu:xray}

We present an interactive learning environment for diagnostic radiography. Our aim is to provide software that can be used to teach radiography using real-time interactive X-ray simulation. Radiographer must master patient positioning and X-ray machine set up in order to reduce unnecessary radiation doses and repetitive acquisitions of X-ray images.
%Se ha presentado un simulador de radiología diagnóstica que permite enseñar y aprender el procedimiento de las proyecciones radiológicas en un entorno seguro. Los técnicos de radiología deben conocer como colocar al paciente y configurar la máquina de rayos X para evitar repeticiones del procedimiento y reducir la dosis de radiación que recibe el paciente.
%innecesaria para el paciente. 
%De esta manera, el simulador proporciona  donde el procedimiento puede ser ensayado sin ningún riesgo para el médico o pacientes. 

Our simulator is based on the proposed algorithm and the \emph{gVirtualXRay} framework. Thanks to the interactive selection pose algorithm and the generation of X-ray images in real-time, the tool allows users to adapt the virtual patient pose while they can observe the radiography. In this way, users are able to select infinite positions for the anatomic model in comparison with the static images provided by books, teaching cases or simulators cited in section \ref{art:entrenamiento}.
%Este simulador esta compuesto por el algoritmo de posicionamiento de pacientes y la librería \emph{gVirtualXRay}. Gracias a la interactividad de la selección de poses del algoritmo y la generación de imágenes en rayos X en tiempo real, se permite adaptar la postura del paciente virtual mientras se visualiza su radiografía. De esta manera, el usuario tiene la posibilidad de seleccionar infinidad de posturas del modelo anatómico interactivamente frente a los métodos clásicos como pueden ser los libros o archivos de casos y a los simuladores introducidos en la sección \ref{art:entrenamiento}.

Additionally, this tool takes advantage from flexibility of the proposed algorithm. It allows using virtual patients from external resources. They must contain at least the skin and bones. The simulator works even if anatomic models are incomplete or mechanical description is not available. As it is expected, X-ray images' quality depends on the quality of the input data. 
%Además, esta herramienta también demuestra la versatilidad del algoritmo presentado en esta tesis. Se puede incorporar cualquier paciente virtual del que se dispongan al menos el tejido óseo y la piel. El algoritmo permitirá animar el modelo anatómico aunque este se encuentre incompleto. Solo la calidad de la imagen radiográfica dependerá de la calidad del modelo de entrada.

%This application has been designed as supplementary material for training and teaching projectional radiography. %Users can practise virtual patient positioning interactively and its internal tissues meanwhile they get an X-ray image in real time. 

%Esta aplicación se ha desarrollado para servir como herramienta complementaria para el aprendizaje y la docencia del procedimiento. % de realizar proyecciones radiológicas. 
%En este simulador, el usuario puede practicar las proyecciones radiológicas seleccionando la postura del paciente virtual interactivamente y sus tejidos internos mientras ve una imagen radiográfica en tiempo real. 

It is important to highlight that the proposed system does not substitute traditional teaching methods but enhance the learning process overcoming their main limitations. Real patient databases and textbooks are the preferred approaches to get theoretical skills, while our simulator allows to interactively perform the main steps of the procedure to train non-cognitive skills. Our tool allows users to reproduce book's projection examples, besides, students can visualise the same virtual patient from different point of views, change the parameters of the X-ray machine, or generate diseases and introduce foreign objects in the anatomic model.
%Es importante destacar que el simulador propuesto no sustituye el método tradicional de enseñanza, sino que mejora el proceso de aprendizaje al proporcionar un entorno interactivo. Frente a los archivos educativos y los libros que proporcionan los conocimientos teóricos, los simuladores permiten practicar el procedimiento interactivamente para conseguir mejorar las habilidades no cognitivas. En este simulador, el usuario puede replicar cualquier proyección que pueda leer de un libro, pero a la vez, se permite interaccionar con el mismo paciente virtual, variar configuraciones de la maquina de rayos X e, incluso, generar enfermedades en el modelo anatómico.

Face and content validity show both the usefulness and suitability of our tool in teaching and/or learning X-ray radiography.  Those results show an interactive learning environment for diagnostic radiography, which can be used to illustrate cases of interest, show specific errors, the effect of the X-ray machine parameters, etc.

%Las validaciones de apariencia y de contenido realizadas (ver sec. \ref{xray:validacion}) permiten confirmar la utilidad como herramienta de entrenamiento. Los resultados muestran una herramienta educativa que puede ser usada como material complementario para los estudiantes de radiología. La aplicación proporciona un entorno seguro e interactivo para comprobar todo tipo de situaciones.

%Este simulador es una unión entre el algoritmo de posicionamiento de pacientes virtuales y la librería de simulación de rayos X. Este algoritmo es susceptible de ser modificado para mejorar las limitaciones que vienen acompañadas del uso de una técnica geométrica.  








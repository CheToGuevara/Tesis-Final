\clearpage
\section{Estructura de la tesis}

Después de este capítulo de introducción, el resto del documento se organizará de la siguiente manera:

En primer lugar, se introducirá en el capítulo \ref{cap:related} la bibliografía necesaria para poner en contexto al lector. Se presentará información relativa a:
\begin{itemize}
    \item Formación de profesionales médicos.
    \item Caracterización de los procedimientos presentes en esta tesis.
    \item Organización del proyecto \acs{RASimAs}.
    \item Revisión bibliográfica sobre las técnicas geométricas y basadas en modelos físicos de la animación de personajes.
\end{itemize}

A continuación, en el capítulo \ref{cap:posing}, se presentará el algoritmo de posicionamiento de pacientes virtuales, el cual es la principal contribución de esta tesis, y los trabajos realizados para la consecución de los objetivos de la tesis. En los siguientes capítulos, se presentarán los casos de uso donde se ha integrado el algoritmo propuesto para demostrar su utilidad. En el capítulo \ref{cap:rasim}, se detallará la integración del algoritmo en la herramienta \acs{ITGVPH} y el desarrollo del módulo \acs{Courseware} en el simulador \acs{RASim}. En el capítulo \ref{cap:xray}, se presentará el desarrollo del simulador de radiología diagnóstica donde se integró el algoritmo propuesto junto con la librería \emph{gVirtualXRay}.

En el capítulo \ref{cap:results}, se exponen los resultados obtenidos por el algoritmo de posicionamiento de pacientes virtuales en cuanto a calidad y rendimiento. Además, se presentarán los problemas ocurridos dentro del proyecto \ac{RASimAs}. Al final, se mostrarán los resultados del simulador de radiología diagnóstica junto con las  validaciones de apariencia y de contenido realizadas. 

Finalmente, en el capítulo \ref{cap:conclu} se resumirán las conclusiones y contribuciones de la presente tesis. Por último, se podrán consultar los apéndices y la bibliografía utilizada.
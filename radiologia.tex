\subsection{Diagnóstico por imagen médica}
\label{art:xraysim}

Dentro de la medicina, hay multitud de técnicas y procesos que generan imágenes médicas (\ac{US}, \ac{TC}, \ac{IRM}, etc.)  que se utilizan para el diagnóstico de enfermedades, afecciones o dolencias. %Existen multitud de técnicas anteriormente mencionadas 
Su utilización depende de la tecnología en las que están basadas, ya que capturan diferentes tejidos o imágenes.
%Esto caracteriza su utilización dependiendo de que tejidos o que imágenes se capturan del cuerpo del paciente. 

El segundo caso de uso que se presenta en esta tesis está orientado al diagnóstico por imagen generada con rayos X. Esta especialidad médica se centra en la generación e interpretación de imágenes que se consiguen al exponer la anatomía objetivo a una radiación electromagnética que será recogida por un detector. Antiguamente, se proyectaba sobre películas fotográficas especialmente preparadas para la fuente emisora. Sin embargo, actualmente la mayoría de las imágenes son almacenadas digitalmente gracias a un detector que permiten guardarlas directamente en un computador.

Se denomina proyecciones radiológicas al procedimiento de situar tanto al paciente como el equipo de radiología para conseguir una imagen de una parte concreta de la anatomía del paciente. Esta técnica define una proyección por cada parte del cuerpo a diagnosticar, requiere por cada procedimiento una postura del paciente y una configuración del equipo de radiografía concreto.

Según el libro \cite{manualpractico}, el procedimiento habitual que se debe seguir se resume en los siguientes pasos:
\begin{enumerate}
    \item Elección de \emph{Bucky}: elemento para reducir la radiación no perpendicular al detector. Este filtro suele utilizarse siempre, excepto en condiciones de emergencia o limitaciones físicas.
    \item Tamaño del chasis y orientación: colocar el detector de la manera que permita cubrir la anatomía a estudiar.
    \item Posición del paciente: teniendo en cuenta las limitaciones del paciente, se le debe posicionar de pie, sentado o decúbito.
    \item Posición de la región anatómica: 
    se colocará al paciente de tal forma que la región objeto de estudio se situé correctamente centrado en el chasis.
    \item Distancia-foco película: se sitúa el emisor de rayos X a la distancia adecuada para la proyección.
    \item Angulación: existen algunas proyecciones donde el chasis se inclina para evitar superposición de estructuras.
    \item Centraje: se centra la proyección de los rayos X en el centro de la región anatómica a estudiar, para lo que es fundamental el conocimiento de la anatomía.
    \item Colimación: se reduce la apertura de la radiación para reducir la exposición del paciente.
    \item Técnica aproximada: se configurará la potencia del equipo de radiología con el objetivo de reducir los niveles de exposición a lo mínimo posible, manteniendo la calidad diagnóstica (principio ALARA del inglés \emph{As Low As Reasonably Achievable} \cite{manualpractico}). 
    \item Indicaciones al paciente: órdenes al paciente en el momento de tomar la imagen.
\end{enumerate}

Al igual que se indica en \cite{manualpractico}, esta metodología, y en concreto los valores de potencia de los emisores de radiación, no es estándar y es susceptible de variaciones dependiendo generalmente del centro de trabajo, profesionales, escuelas, rendimiento y antigüedad de los equipos, etc. 



%https://www.bluephantom.com/product/Sciatic-Nerve-Regional-Anesthesia-Ultrasound-Training-Model.aspx?cid=428
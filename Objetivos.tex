\section{Objetivos}
\label{intro:objetivos}
%\todo{Cuida más los objetivos. Mas o menos se entienden pero tienen que sonar claros y rigurosos.}
%Con la  cuando la posición del paciente en el momento que se adquiere una imagen médica (por ejemplo \ac{IRM} o \ac{TC}) es normalmente diferente a la posición requerida por el procedimiento médico que se este simulando. A la vez, estas imágenes suelen ser locales y no son capaces de recuperar cada uno de los tejidos del paciente o su comportamiento mecánico.

Con la hipótesis de partida enunciada, a continuación, se presentan los objetivos principales que se pretenden alcanzar a lo largo de la presente tesis:

%\todo{Asegurar que todo esta escrito en futuro}
\begin{itemize}
\item Diseño de un algoritmo para el posicionamiento de modelos anatómicos, desde la posición en la que fueron modelados hasta cualquier posición requerida.
\\
%\todo{no puede ir en pasado son objetivos!!!!. Escríbelos como tales}

Se propondrá una técnica que permita transformar los modelos anatómicos de pacientes virtuales, los cuales originalmente se encuentran en una postura diferente a la posición necesaria para el entrenamiento de un determinado procedimiento médico. %\todo{desde la coma suana raro}. 
La técnica propuesta será capaz de modificar un modelo anatómico con estructuras internas de manera interactiva, siendo esto posible, aunque el modelo no fuera completo o no estuvieran disponibles las propiedades mecánicas de los tejidos. En resumen, la técnica que se proponga deberá verificar los siguientes requisitos:
%\todo{sin necesidad de disponer de una descripción mecánica de las porpiedades de los tejidos y incluso sin disponer de un modelo completo, es decir, no es necesario que estén modeladas todas las estructuras anatómicas. (Pon esto bonito)}

\begin{itemize}
%\todo{Yo aquí subdividirá el objetivo 1 en los requisitos de la técnica}
    \item Debe funcionar en tasas interactivas.
    \item Debe poder trabajar con modelos incompletos.
    \item No necesita las descripciones mecánicas de los tejidos.
    
\end{itemize}

%\todo{Tenemos que hablar.  2. La idea es probar el algortimo en dos casos de uso reales. Uno que demuestre que funciona con información incompleta y otro que demuestre que funciona en aplicaciones interactivas. Ojo con decir que probamos que la transferencia de conocimientos es adecuada, no lo hacemos!!!!. Tienes que insinuarlo sin dejarlo explicito.

\item 	Validación del algoritmo. \\
%\todo{Yo diría en los objetivos que la técnica servirá para entrenar si se al incorporarla en simuladores estos pueden usarse para entrenar.}
%\todo{Los objetivos son muy importantes. Redacta esto mejor. }
Con el fin de demostrar que el algoritmo propuesto pueda ser utilizado en el entrenamiento médico, este será incorporado en dos tipos de aplicaciones de \ac{RV}. En el primer caso, se intentará demostrar que el algoritmo puede adaptar la postura de un modelo anatómico en un escenario donde no se dispone de modelos completos desde un punto de vista anatómico y en los que tampoco se disponga de los parámetros que caracterizan el comportamiento mecánico de las estructuras presentes. En el segundo caso de uso, el algoritmo mostrará su utilidad en un entorno donde el usuario deberá modificar interactivamente la posición del paciente como parte del procedimiento. 

%Objetivos secunadrios:

%- Objetivos del algoritmo

%- Objetivos de los dos casos de uso.

%--- Recureda. En el simuladador de anestesia regional no se cambia la pose en tiempo real (validamos descripciones incompletas)

%--- El de radiología la selección de la pose es parte del comportamiento, aunque por otro lado hay menos estructuras anatomicas relevantes.

\begin{itemize}
    \item 	Caso de uso: Simulador de anestesia regional y herramienta \ac{ITGVPH}
    %\todo{No me gusta que lo vendas como offline. Hemos vendio que un experto valida la pose por lo que la herramienta itne que ser interactiva.}:
    
    Un objetivo del proyecto \ac{RASimAs} es la creación de la herramienta \ac{ITGVPH} para generar una base de datos con pacientes virtuales que representen distintas variaciones anatómica promedio y que posteriormente será utilizada en \ac{RASim}.
    El algoritmo propuesto se integrará y comunicará con otros módulos que componen la herramienta, permitiendo animar y adaptar los \ac{VPH} generados. El método estará en comunicación con otros módulos desarrollados dentro del proyecto \ac{RASimAs}. Además, se desarrollarán los módulos del simulador con la finalidad de comprobar si la generación de estos \ac{VPH} es útil para entrenar el procedimiento de \ac{RA}.
    
   % \del{Además, se desarrollará un \emph{software} (\acs{Courseware}) para el simulador \ac{RASim} que se encargará de la comunicación con los demás módulos, y gestionará la interacción del usuario en una plataforma de entrenamiento auto guiada.}\todo{El courseware no es un objetivo. Lo que haces es desarrollar modulos del simulador de cara que este pueda utilizarse para entrenar a pacientes. No des muchos más detalles.} 

%\todo{Idea, no se necesitan pacientes reales sino pacientes medios. No se necesita una deformación presica, sino plausible}

%\todo{No entiendo este segundo parrafo. Puedes hablar del curseware. No se porque hablas de los hapticos y del Hw eso va antes?????}
    

    \item Caso de uso: Simulador de radiología diagnóstica

%\todo{Lo que tienes que vender aquí es que el posing del paciente es parte integral del procedimeinto y que se requiere un algorimo robusto que funcione en con tasas de refresco interactivas.}
%En este caso de uso, el algoritmo propuesto permite al usuario modificar la postura del paciente con el objetivo de entrenar el procedimiento de radiología diagnóstica.\todo{no has dicho nada} 
En radiología diagnóstica, el médico debe posicionar al paciente y configurar la máquina de rayo X de manera que la región anatómica objeto de estudio sea adecuadamente capturada. El algoritmo propuesto permite modificar la posición de un paciente virtual interactivamente, permitiendo probar distintas proyecciones y obtener imágenes de rayos X simultáneamente. El simulador permitirá entrenar las proyecciones radiográficas sin ningún riesgo para el paciente o el usuario. Este simulador, podrá ser usado tanto por profesores como por estudiantes de radiología como una herramienta adicional a las técnicas clásicas de aprendizaje.
\end{itemize}

\end{itemize}

% \begin{itemize}
%     \item 	Diseño del entrenamiento y evaluación del simulador.
% \end{itemize}

% La \ac{RV} proporciona una herramienta muy útil para poder guiar y ayudar en el proceso de aprendizaje al usuario. Mediante tareas guiadas, el usuario podrá aprender y perfeccionar el procedimiento quirúrgico, que ayudará a suavizar así la curva de aprendizaje y reducir posibles riesgos en el futuro. El simulador permitirá empezar por los caso más sencillos e ir facilitando el entrenamiento de casos más complejos. El objetivo es, que el usuario aprenda desde las competencias más básicas hasta simulaciones de situaciones reales, intentando minimizar el tiempo de aprendizaje. El usuario puede repetir el entrenamiento hasta que adquiera las habilidades requeridas. Debido a que cada vez más están presentes los simuladores en el currículum de los estudiantes de medicina, es necesario  crear unas métricas que sean útiles para evaluar a los estudiantes de manera cualitativa.




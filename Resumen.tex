\thispagestyle{empty}
\chapter*{Resumen\markboth{Resumen}{Resumen}}


En el contexto del entrenamiento médico, los simuladores de realidad virtual permiten desarrollar las habilidades cognitivas y no cognitivas necesarias para la práctica de un determinado procedimiento médico. Estos proporcionan un entorno seguro, repetible y variado, donde el profesional sanitario en formación se enfrente a la mayor cantidad de casos y variaciones anatómicas posibles.

%\todo{en pasado o presente?, pasado esta bien.}
Esta tesis ha sido desarrollada dentro del contexto del proyecto \emph{RASimAs}, financiado por el 7º Programa del Marco de la Unión Europea. El objetivo principal es el desarrollo de herramientas que faciliten el entrenamiento y la práctica de la anestesia regional. Para cumplir con este objetivo se ha propuesto dos sistemas: un entrenador de realidad virtual llamado \emph{Regional Anaesthesia Simulator (RASim)} y un asistente en quirófano llamado \emph{Regional Anaesthesia Assistant (RAAs)}.
Con el objetivo de enfrentar a los médicos a la mayor cantidad de variabilidad anatómica posible, en este proyecto se ha propuesto desarrollar un entorno integrado de generación de pacientes virtuales, que permita crear una base de datos de modelos anatómicos que pueda ser utilizado por el simulador \emph{RASim}. Esta herramienta genera pacientes virtuales (VPH), a partir del registro de un modelo virtual con imágenes de pacientes reales. Además, es necesario transformar la postura del paciente generado a las diferentes posiciones requeridas por el procedimiento de anestesia regional. En concreto, esta tarea ha sido designada al grupo \emph{GMRV} de la \emph{Universidad Rey Juan Carlos}, estableciendo la motivación principal de esta tesis.

A lo largo de este trabajo, se presenta una nueva técnica que permite transformar los modelos anatómicos de pacientes virtuales de su postura original a la posición necesaria por el procedimiento médico. Esto es posible aunque los modelos anatómicos se encuentren incompletos o falten sus descripciones mecánicas. Además, ya que el usuario supervisará la deformación que se aplicará al paciente virtual, el sistema debe tener tasas de refresco interactivas. Para cumplir con estos requisitos, se ha desarrollado una técnica geométrica basada en el cauce de la animación esqueletal, en lugar de utilizar un método basado en física debido a que estos últimos presentan una serie de problemas. En estos casos, se requiere caracterizar mecánicamente los tejidos que se van a simular los cuales no siempre se encuentran disponibles. Además, los métodos basados en modelos físicos se centran en conseguir deformaciones precisas, resultando en un alto grado de complejidad que impide conseguir tasas de refresco interactivas. Frente a los métodos basados en física, se ha optado por utilizar una técnica geométrica que proporciona soluciones plausibles que el usuario pueda interpretar como reales.

Partiendo de la piel y el tejido óseo del paciente virtual, se genera un campo de desplazamientos continuo en el interior del paciente virtual que se utiliza para transformar sus estructuras internas. Las operaciones más costosas se han delegado a un proceso previo que genera toda la información necesaria para que el usuario pueda seleccionar la postura del paciente virtual interactivamente. Además, se ha propuesto un refinamiento opcional basado en un método físico que intenta conservar el volumen del modelo anatómico. Con el objetivo de validar la hipótesis por la cual un algoritmo geométrico puede generar nuevas posturas de un paciente virtual junto con sus tejidos internos para ser utilizadas en el contexto del entrenamiento de un procedimiento médico, se han propuesto dos casos de uso. 

En primer lugar, el algoritmo propuesto se ha integrado en el entorno de generación de pacientes virtuales, permitiendo animar y adaptar al profesional médico los modelos anatómicos generados por la suite. Se intenta demostrar que el algoritmo puede adaptar la postura de un modelo anatómico en un escenario donde no se dispone de modelos completos y no se dispongan de sus propiedades mecánicas. Además, con la finalidad de comprobar si los pacientes virtuales son útiles para  el entrenamiento del procedimiento de RA, se ha contribuido en la creación del módulo Courseware. Esta plataforma de aprendizaje donde el usuario podrá practicar y desarrollar sus habilidades no cognitivas, gestiona todos los componentes del simulador y se encarga de implementar las tareas de entrenamiento. Por problemas de precisión de los dispositivos hápticos, no se ha podido realizar una evaluación clínica del simulador. 



En segundo lugar, se ha desarrollado un simulador de radiología diagnóstica gracias a la librería \emph{gVirtualXRay} en colaboración con \emph{Dr. Franck P.Vidal}. En este procedimiento, el médico debe posicionar al paciente y configurar la máquina de rayos X de manera que la región anatómica objetivo sea adecuadamente capturada. El algoritmo propuesto demuestra su capacidad para transformar la postura de un paciente virtual interactivamente, y así probar distintas proyecciones, mientras que la librería \emph{gVirtualXRay} permite  obtener imágenes de rayos X simultáneamente. Se ha realizado una encuesta a especialistas en radiología para comprobar su validez aparente y de contenido, donde se ha preguntado acerca de su opinión sobre el realismo y la utilidad del simulador. Los resultados obtenidos confirman su beneficio como herramienta adicional a las técnicas clásicas de aprendizaje.



%%% 
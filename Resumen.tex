\thispagestyle{empty}
\chapter*{Resumen\markboth{Resumen}{Resumen}}



En el contexto del entrenamiento médico, los simuladores de realidad virtual permiten desarrollar las habilidades cognitivas y no cognitivas necesarias para la práctica de un determinado procedimiento médico. Proporcionan un entorno seguro, repetible y variado, donde el profesional sanitario en formación se puede enfrentar a la mayor cantidad de casos y variaciones anatómicas posibles.

En la literatura se puede encontrar una gran diversidad de simuladores médicos. %\todo{se puede unir con la frase anterior? Esta puesta a proposito}
En los últimos años, están tomando una mayor importancia en el currículum de los estudiantes, tanto en el campo de la cirugía como en el de otras especialidades \cite{PATEL2017266.e7}. Habitualmente, los simuladores son específicos de cada procedimiento médico, como por ejemplo en \cite{cecil2017advanced} donde se presenta un simulador de cirugía ortopédica. Es también muy habitual encontrar simuladores de cirugía cardiovascular como \cite{korzeniowski2018vcsim3}. En algunos casos se combinan tecnologías de dos o más especialidades para realizar un procedimiento concreto. En  \cite{villard2014interventional} se presenta un simulador de radiología intervencional donde se entrena la habilidad quirúrgica. Estos y otros simuladores están empezando a reemplazar a las técnicas tradicionales de enseñanza como: la práctica supervisada sobre pacientes reales (\emph{by doing}), la utilización de \emph{fantomas} \cite{phantomra} o la práctica en cadáveres \cite{Tsui2007}.

%\todo{en pasado o presente?, pasado esta bien.}
Esta tesis ha sido desarrollada dentro del contexto del proyecto \emph{RASimAs}, financiado por el 7º Programa del Marco de la Unión Europea. El objetivo principal es el desarrollo de herramientas que faciliten el entrenamiento y la práctica de la anestesia regional. Para cumplir con este objetivo se han propuesto dos sistemas: un entrenador de realidad virtual llamado \emph{Regional Anaesthesia Simulator (RASim)} y un asistente en quirófano llamado \emph{Regional Anaesthesia Assistant (RAAs)}.
Con el objetivo de enfrentar a los médicos a la mayor cantidad de variabilidad anatómica posible, en este proyecto se ha propuesto desarrollar un entorno integrado de generación de pacientes virtuales (\emph{ITGVPH}) que permita crear una base de datos de modelos anatómicos que puedan ser utilizados por el simulador \emph{RASim}. Esta herramienta genera pacientes virtuales (\emph{VPH}), a partir del registro de un modelo virtual con imágenes de pacientes reales. Además, resulta necesario transformar la postura del paciente, lo cual permite generar las diferentes posiciones requeridas por el procedimiento de anestesia regional. Esta tarea, en concreto, ha sido designada al grupo \emph{GMRV} de la \emph{Universidad Rey Juan Carlos}, estableciendo la motivación principal de esta tesis.

El objetivo principal en este trabajo de tesis doctoral es el diseño de un algoritmo para el posicionamiento de modelos anatómicos, desde la posición en la que han sido modelados hasta cualquier posición requerida. Con el fin de demostrar que el algoritmo propuesto puede ser utilizado en el entrenamiento médico, este será incorporado en dos tipos de aplicaciones de realidad virtual.

La metodología empleada en este trabajo incluye el desarrollo de un algoritmo geométrico. Se presenta una nueva técnica que permite transformar los modelos anatómicos de pacientes virtuales de su postura original a la posición necesaria por el procedimiento médico. Esto es posible aunque los modelos anatómicos se encuentren incompletos o falten sus descripciones mecánicas. Además, como el usuario debe supervisar la deformación que se aplicará al paciente virtual, el sistema debe tener tasas de refresco interactivas. Para cumplir con estos requisitos, se ha desarrollado una técnica geométrica basada en el cauce de la animación esqueletal, en lugar de utilizar un método basado en modelos físicos debido a que estos últimos presentan una serie de problemas. En estos casos, se requeriría caracterizar mecánicamente los tejidos que se van a simular, los cuales no siempre se encuentran disponibles. Además, los métodos basados en modelos físicos se centran en conseguir deformaciones precisas, resultando en un alto grado de complejidad, que impide conseguir tasas de refresco interactivas. Frente a los métodos basados en modelos físicos, se ha optado por utilizar una técnica geométrica que proporciona soluciones plausibles en tiempos interactivos y que el usuario puede interpretar como reales.

El método propuesto utiliza los modelos de la piel y el tejido óseo del paciente virtual para generar un campo de desplazamientos continuo en el interior del modelo, que se utiliza para transformar sus estructuras internas. Las operaciones más costosas se han delegado a un proceso previo que genera toda la información necesaria para que el usuario pueda seleccionar la postura del paciente virtual interactivamente. Además, se ha propuesto un refinamiento opcional basado en un método físico que intenta conservar el volumen del modelo anatómico. Con el objetivo de validar la hipótesis por la cual un algoritmo geométrico puede generar nuevas posturas de un paciente virtual junto con sus tejidos internos para ser utilizadas en el contexto del entrenamiento de un procedimiento médico, se han propuesto dos casos de uso. 

En primer lugar, el algoritmo propuesto se ha integrado en el entorno de generación de pacientes virtuales (\emph{ITGVPH}) de \emph{RASimAs}, permitiendo al profesional médico animar y adaptar los modelos anatómicos generados por la \emph{suite}. Se intenta demostrar que el algoritmo puede adaptar la postura de un modelo anatómico en un escenario donde no se dispone de modelos completos ni de sus propiedades mecánicas. Además, con la finalidad de comprobar si los pacientes virtuales son útiles para el entrenamiento del procedimiento de RA, se ha contribuido en la creación del módulo Courseware. Esta plataforma de aprendizaje, donde el usuario podrá practicar y desarrollar sus habilidades no cognitivas, gestiona todos los componentes del simulador y se encarga de implementar las tareas de entrenamiento. Por problemas de precisión de los dispositivos hápticos, no se ha podido realizar una evaluación clínica del simulador. 

En segundo lugar, se ha desarrollado un simulador de radiología diagnóstica gracias a la librería \emph{gVirtualXRay} en colaboración con el \emph{Dr. Franck P.Vidal}. En este procedimiento, el médico debe posicionar al paciente y configurar la máquina de rayos X de manera que la región anatómica objetivo sea adecuadamente capturada. El algoritmo propuesto demuestra su capacidad para transformar la postura de un paciente virtual interactivamente, y así probar distintas proyecciones, mientras que la librería \emph{gVirtualXRay} permite obtener imágenes de rayos X simultáneamente. Se ha realizado una encuesta entre especialistas en radiología para comprobar su validez aparente y de contenido, en la que se ha preguntado acerca de su opinión sobre el realismo y la utilidad del simulador.

Los resultados obtenidos muestran que la técnica de posicionamiento de pacientes virtuales se puede incorporar en cualquier simulador que requiera cambiar la pose de un determinado modelo anatómico con estructuras internas en tiempo real. Además, permite adaptar cualquier tejido a la pose deseada, de forma independiente al resto de tejidos, pudiendo así trabajar con modelos incompletos.

Para finalizar, las conclusiones de este trabajo de tesis se pueden resumir en la creación de un cauce de animación esqueletal semiautomático que permite transformar un modelo anatómico de una postura a la posición requerida, en este caso, en el entrenamiento de un determinado procedimiento médico. Este algoritmo ha sido incorporado en dos aplicaciones con objetivos distintos. Por una parte, se ha incorporado en la \emph{suite} \emph{ITGVPH} que permite crear una base de datos de pacientes virtuales. Aunque esta aplicación no ha podido ser evaluada, los resultados obtenidos por la técnica propuesta han sido favorablemente valorados por los socios médicos del proyecto. Por otra parte se ha incorporado en simulador de radiología diagnóstica. En este caso, los resultados obtenidos en las validaciones confirman su beneficio como herramienta adicional a las técnicas clásicas de aprendizaje.



%%% 
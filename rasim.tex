\section{RASim}
\label{rasim:rasim}
El simulador \ac{RASim} proporciona el entorno de trabajo para el entrenamiento de la \ac{RA}.  
Los \ac{WP} 4 y 5 definen todas las tareas para el desarrollo de los prototipos \ac{RASim} y \ac{RAAs}. Las tareas del 1 al 4 del \ac{WP} 4 corresponden a los módulos por separado de \ac{RASim} y las tareas 1 y 2 del \ac{WP} 5 corresponden a su integración en el prototipo. 
La responsabilidad del primer prototipo de \ac{RASim} recae en el equipo de trabajo de \ac{SG}, siendo cada módulo desarrollado por diferentes integrantes del proyecto. Aun así, la comunicación entre los diferentes grupos ha sido fundamental para conseguir una buena cohesión de todos los elementos del simulador. En esta tesis, se ha contribuido al simulador con el módulo de \ac{Courseware}. Este módulo se encarga del entrenamiento y gestiona los componentes del simulador. Por tanto, se trabajó conjuntamente con los demás participantes de \ac{RASimAs} en la construcción del prototipo \ac{RASim}.


%El objetivo era realizar un entorno de trabajo que permite al usuario entrenar el procedimiento de la forma más parecida a la situación real que se encontrarán en su futuro profesional.

% \subsection{Caso de uso}
% \label{rasim:casodeusorasim}

 

%Por ello el usuario, normalmente estudiante de anestesiología, se sentará delante del entorno de trabajo y se identificará en el sistema para poder almacenar y seguir su evolución durante todo el proceso. En las primeras sesiones, el usuario podrá revisar contenido multimedia para familiarizarse con el simulador y el procedimiento médico. El siguiente paso, será realizar una sesión guiada \del{en la cual} que  consiste en que el simulador va orientando al usuario en todos los pasos que debe realizar para completar con éxito el procedimiento médico. Durante el proceso, el usuario recibirá retroalimentación formativa por los errores o aciertos cometidos. Al finalizar la sesión, el simulador mostrará una retroalimentación sumativa sobre el desempeño del usuario y métricas recogidas por el simulador. Cada estudiante tiene designado un tiempo y objetivos mínimos que deberá cumplir antes de continuar con el entrenamiento. Una vez completada la primera sesión guiada, el usuario podrá repetir la sesión guiada o realizar una sesión libre, en la que el simulador no proporcionará ninguna indicación y sólo se recogerán métricas de la sesión. Todas las evaluaciones y métrica serán enviadas a un servidor dónde el supervisor podrá consultar y revisar la evolución de los usuarios.

El simulador consta de una mesa de trabajo donde se situarán todos los periféricos con los que interaccionará el usuario. En la mesa se encontrarán dos monitores para mostrar la escena virtual y la imagen de \ac{US}, junto con los dispositivos que simulan una sonda de ultrasonidos (\emph{\acs{tracker}}) y una aguja (dispositivo háptico). En cuanto al software, en la figura \ref{fig:coursearq} se puede observar la arquitectura software propuesta para el simulador: 
\begin{itemize}

    \item Módulo US: se encarga de simular el proceso de generación de la imagen \ac{US}.
    \item \acs{SOFA}: es el módulo encargado de la simulación física.
    \item \emph{H3D}: es un módulo de visualización de la escena virtual y comunicación con los dispositivos.
    \item \acs{Courseware}: es la plataforma de aprendizaje que se comunica con el módulo \emph{H3D}, \ac{US} y el servidor local .
    \item Servidor local y global: es un módulo para almacenar y gestionar la información del simulador.
\end{itemize}

\begin{figure}[ht]
    \centering
    \includegraphics[width=\textwidth]{IMG/RasimArq.pdf}
    \caption{Descripción inicial de la arquitectura de \acs{RASim}.}
    \label{fig:coursearq}
\end{figure}
%\todo{rehago la imagen supongo}


A continuación se describirán brevemente los distintos componentes del simulador.



%Estos dispositivos están conectados a un computador. El computador centraliza el software desarrollado y la interacción con el usuario. A través de dos monitores se mostrará al usuario información visual y a través de un dispositivo háptico se proporcionará respuesta táctil. Además, se incorporará un \ac{Courseware}\footnote{Término ampliamente utilizado en la industria } que servirá para proporcionar una plataforma de aprendizaje y autoevaluación. Al mismo tiempo, esta herramienta puede recopilar métricas y mostrar información útil al usuario. Esta aplicación será la encargada de cargar diferentes escenarios, guardar sesiones de entrenamiento, mostrar tareas y lecciones específicas del entrenamiento.

%El prototipo creado tendrá que ser evaluado como parte del proyecto europeo en estudios clínicos controlados dirigido tanto a expertos como estudiantes de anestesiología. Antes de alcanzar los experimentos clínicos, cada módulo del simulador debe ser probado por separado (\ac{WP}4 sección \ref{art:rasimas}) para asegurar el proceso de evaluación y permitir un desarrollo iterativo con el objetivo de construir un prototipo validado y fiable (\ac{WP}5 sección \ref{art:rasimas}). 

% Para cumplir con ello se han propuesto las siguientes especificaciones para el simulador:

% \begin{itemize}
%     \item Computador: el sistema donde se ejecutará el sistema deberá ser lo suficiente potente como para ejecutar simulaciones físicas que permiten simular el comportamiento de los tejidos, calcular la retroalimentación háptica y generar la representación visual sin que afecte a la interactividad del usuario.
%     \item Plataforma de trabajo: para estandarizar el entorno de trabajo se ha diseñado una plataforma que permite conocer la disposición tanto del maniquí como de los dispositivos \ac{E/S} con el objetivo de ser fácilmente reproducible en diferentes instituciones. Deberá estar diseñado para personas diestras y zurdas.
%     \item Maniquí y región de interés: con el objetivo de proporcionar el mayor realismo posible, se ha propuesto incluir un maniquí en la plataforma para que el usuario interactúe sobre un espacio de trabajo controlado.
%     \item Monitores: es necesario mostrar información visual sobre la habitación virtual y la imagen de \ac{US}. Además, se mostrará información útil para el aprendizaje.
%     \item Sonda de ultrasonidos: es uno de los instrumentos principales de la \ac{RA}, por ello conviene que la sonda sea lo más realista posible.
%     \item Aguja: instrumento fundamental en el procedimiento. Al ser una parte importante del entrenamiento, el simulador captura su movimiento y posición y devuelve fuerzas que simularan las sensaciones para el usuario como si fuera en un entorno real.
%     \item \ac{Courseware}: una aplicación que permita integrar todos los elementos del sistema además de proporcionar un sistema de entrenamiento y evaluación.
% \end{itemize}



\subsection{Hardware}


\subsubsection{Mesa de trabajo}

El principal objetivo de crear una mesa de trabajo es estandarizar la posición relativa de los dispositivos anteriormente citados. %Además de suplir las carencias de estos para mejorar la inmersión del usuario en el simulador.
Se ha diseñado una plataforma en forma de "T" para poder incorporar de manera perpendicular el campo de trabajo del dispositivo magnético por una parte y el dispositivo háptico por otra. De esta manera, el dispositivo háptico puede ser intercambiado de posición para permitir que el simulador pueda ser configurado para que se pueda realizar con ambas manos. Situado en el centro, se colocará un maniquí de un material deformable para simular la zona anatómica del paciente. 

Junto con la plataforma, se encuentran los monitores situados de manera frontal. Esto hace que el usuario tenga todos los elementos en su rango de visión y evite realizar una rotación de cuello. En el primero, se muestra la vista del paciente que consiste en la \emph{renderización} de la escena virtual dónde se encuentra el modelo anatómico de un paciente y una representación virtual de los instrumentos médicos. Además, esta vista se puede utilizar para añadir elementos virtuales que mejoren la efectividad del entrenamiento. 
Por otra parte, en el segundo monitor se muestra el software de entrenamiento en el que se incluye la imagen de \ac{US}. Una versión del prototipo se puede ver en la figura \ref{fig:simulator}.


\begin{figure}[ht]
    \centering
    \includegraphics[width=0.95\textwidth]{IMG/simulator.jpg}
    \caption{Diseño de la mesa de trabajo. El rectángulo de espuma se sitúa en el centro. A continuación, el emisor del dispositivo magnético, y los monitores. A los laterales, el dispositivo háptico puede ser intercambiado de posición.}
    \label{fig:simulator}
\end{figure}


\subsubsection{Maniquí} 
En el procedimiento de \ac{RA}, el radiólogo apoya la sonda de ultrasonidos y parte del brazo en la anatomía del paciente. Por tanto, es necesario reproducir la zona anatómica que ayudará al usuario a apoyar la sonda de ultrasonidos e introducir la aguja en algún lugar de manera segura y controlada. Esto requiere que el material sea resistente tanto a presión como a la inserción de la aguja repetidamente sin introducir fuerzas adicionales al sistema. Por tanto, se diseñó un rectángulo de espuma de poliuretano que sirviera de recipiente para un relleno con espuma floral como se puede observar en la figura \ref{fig:simulator}.

% La idea original era obtener un reemplazo rápido porque la espuma floral, aunque permitía una inserción de la aguja sin apenas resistencia, era muy débil ante presión y rotaciones de la aguja. Además, las inserciones continuas debilitaban el material y pueden dar información a los usuarios que utilicen el simulador posteriormente.
% Para añadir consistencia a la espuma floral se ideó añadir una capa de silicona por encima. Esta capa permitía que la espuma no se deshiciera con el rozamiento y además proporcionaba cierta estabilidad, pero a la vez introducía un pequeño rozamiento que habría que salvar al introducir una aguja. En la figura \ref{fig:silicona} se puede observar un prototipo.

% \begin{figure}[ht]
%     \centering
%     \includegraphics[width=0.5\textwidth]{IMG/silicona.jpg}
%     \caption{Espuma floral cubierta de una capa de silicona para mejorar su consistencia.}
%     \label{fig:silicona}
% \end{figure}

% Desde la \ac{URJC} se propuso utilizar otro material que no implique un bloque de espuma. Existe un material arenoso que intenta simular un fluido \emph{no newtoniano} (ver figura: \ref{fig:arena}) de tal forma que es resistente a la presión y rozamiento de la sonda, pero permite una inserción de la aguja sin apenas resistencia. Además, al ser un material con características parecidas a las de un fluido, las continuas inserciones no debilitan el material. Puede ser fácilmente reemplazable y su coste es pequeño.


% \begin{figure}[ht]
%     \centering
%     \includegraphics[width=0.5\textwidth]{IMG/arena.jpg}
%     \caption{Arena \emph{kinética}}
%     \label{fig:arena}
% \end{figure}
% \todo{rehacer figura arena}

% Existen otros problemas al restringir el espacio de trabajo a un rectángulo de espuma o arena. Este rectángulo no representa fielmente el modelo anatómico. Al no ser iguales que el modelo del paciente virtual, esto generaba inconsistencias entre el mundo virtual y el real que afectaba a la realización correcta del procedimiento.
% Actualmente, la tecnología de impresión 3D permite una solución rápida y barata para fabricar modelos personalizados. Con el objetivo de mejorar el realismo de la aplicación, se ha diseñado el rectángulo contenedor que tuviera la forma anatómica utilizando el mismo modelo que se usaría en el simulador. Solo haría falta introducir y adaptar la espuma o arena en el interior de este modelo. Esto permitiría utilizar la forma anatómica como se puede observar en la figura \ref{fig:pierna}. 

% \begin{figure}[ht]
%     \centering
%     \includegraphics[width=0.5\textwidth]{IMG/pierna.PNG}
%     \caption{Versión del contenedor de la espuma con forma de pierna.}
%     \label{fig:pierna}
% \end{figure}

% Aun así, esta solución depende enormemente del modelo anatómico que se esta simulando y necesitaría imprimir un soporte para cada modelo virtual que se utilizará en el simulador. 



% \subsubsection{Monitores}

% Una característica importante de una herramienta de \ac{RV} es la inmersión del usuario en el sistema. Esta inmersión está normalmente muy influenciada por los dispositivos de entrada y salida que le conecta con el motor de \ac{RV}. Cuanto más se asemejen a la realidad, proporcionarán una sensación de inmersión y presencia que ayudará al usuario a mejorar su experiencia en el sistema y por tanto, el objetivo que hubiera detrás de éste (entrenamiento, preparación, etc...). Normalmente, como se ha introducido en el capítulo \ref{cap:intro}, estos dispositivos están orientados a determinados canales sensoriales. En este sistema, sólo se ha considerado el canal visual y el táctil debido a que el procedimiento de \ac{RA} no implica otro tipo de sentidos. A continuación, se hará una descripción de los dispositivos que componen el prototipo desarrollado en el proyecto.


% \paragraph{Vistas del paciente y del ecógrafo}\mbox{}\\

% La vista es el sentido de entrada dominante en muchos aspectos cotidianos y en este simulador, concretamente, es necesario que el usuario pueda situar los instrumentos médicos en relación con la anatomía del paciente. Además, necesita recabar toda la información posible de los tejidos internos que le proporciona la imagen de ultrasonidos y no puede observarse a simple vista. Por tanto, el usuario utiliza su capacidad visual para colocar los instrumentos médicos en la posición requerida y a la vez requiere del ecógrafo para su correcto desempeño.

% Para representar la vista del paciente y de la máquina de ultrasonidos, se ha incorporado dos pantallas al entorno de trabajo del simulador. 
% En uno de ellos se muestra la vista del paciente. Esta consiste en la \emph{renderización} de la escena virtual dónde se encuentra el modelo anatómico de un paciente y una representación virtual de los instrumentos médicos. Además, esta vista se puede utilizar para añadir elementos virtuales que mejoren la efectividad del entrenamiento. 
% Por otra parte, en el otro monitor se utiliza para mostrar el software de entrenamiento que incluye la imagen de ultrasonidos generada por el motor de \ac{RV}.

\subsubsection{Sonda de ultrasonidos}

El posicionamiento de la sonda de ultrasonidos es clave en el procedimiento de \ac{RA}. La técnica eco-gráfica es fundamental para encontrar el nervio a anestesiar. Con el objetivo de simular este aparato médico, se ha diseñado un modelo impreso en 3D que imita una sonda real que envuelve al sensor del dispositivo de seguimiento. Para esta finalidad, se ha utilizado un \emph{\acs{tracker}} magnético\footnote{Ascension 3D-Guidance trakSTAR, módelo 800 \cite{Ascension}.} como se puede observar en la figura \ref{fig:simulator}. 
%Este dispositivo es capaz de saber la posición de un receptor o sensor gracias a la generación de un campo magnético producido por un emisor. La deformación del campo magnético producido por el receptor ayuda al dispositivo a conocer su posición y orientación. Este sensor es el que se introduce dentro del modelo impreso y sirve para saber la posición de la sonda respecto a la mesa de trabajo. %De esta manera el usuario utiliza un instrumento parecido lo más posible a uno real.

% \begin{figure}[ht]
%     \centering
%     \includegraphics[scale=0.65]{IMG/tracker.png}
%     \caption{\emph{Ascension 3D-Guidance trakSTAR, módelo 800} \cite{Ascension} utilizado en \ac{RASim}}
%     \label{fig:tracker}
% \end{figure}

% En un principio, el diseño del modelo 3D intenta recrear la sonda original a la vez que se incluye el dispositivo de seguimiento dentro de él. Se ha diseñado un modelo compuesto por dos piezas que en el interior acoge el sensor sin que el usuario sea consciente (ver figura \ref{fig:m3d}). Además, el propio cable del sensor es similar al cable que se encontrarán los usuarios en un entorno real.

% \begin{figure}[ht]
%     \centering
%     \includegraphics[width=0.5\textwidth]{IMG/modelo3d.png}
%     \caption{Primer prototipo del modelo impreso de la sonda de ultrasonidos para alojar el sensor del \emph{tracker} magnético.}
%     \label{fig:m3d}
% \end{figure}

% Debido a limitaciones de hardware de poder recrear fielmente la deformación de los tejidos de la piel cuando el médico presiona la sonda contra ella, se decidió junto con los profesionales médicos del proyecto que habría que incorporar algún mecanismo que ofreciera al usuario una sensación de elasticidad de la piel. Así de esta manera, se ideó una pieza adicional conectada al modelo 3D a través de un muelle que permitía la sensación de tejido flexible al usuario como se puede observar en la figura \ref{fig:m3dv2}.

% \begin{figure}[ht]
%     \centering
%     \includegraphics[width=0.5\textwidth]{IMG/modelo3dv2.jpg}
%     \caption{Modelo impreso de la sonda de ultrasonidos con la actualización que permite tener sensación de elasticidad de la piel }
%     \label{fig:m3dv2}
% \end{figure}


\subsubsection{Aguja}
\label{rasim:aguja}
En el procedimiento de \ac{RA} guiado por \ac{US}, el primer paso es la localización del nervio. Una vez localizado, la inserción de la aguja y su aproximación al nervio que se va a anestesiar es otro punto crucial del procedimiento. El médico debe introducir la aguja en una posición relativa a la sonda, por tanto, es necesario que el usuario practique con los dispositivos. Se mueve la sonda y cuando el usuario tiene el nervio localizado, se introduce la aguja de forma segura en el maniquí. El simulador debe interpretar la inserción de la aguja y devolver las fuerzas correspondientes a atravesar las fascias que se encuentran en la anatomía. 

%El medico utiliza el taco para saber donde esta la aguja. Siempre que se traviesan las fascias el lo siente.

La inserción real de la aguja implica que su movimiento pueda definirse a través de 5 \ac{DOF}. La correcta posición, el movimiento adecuado y la sensación de la aguja son muy importantes en el entrenamiento de los nuevos médicos. Por ello la selección del dispositivo con el que replicar estos comportamientos no es sencillo. Los \emph{\acs{tracker}s} magnéticos anteriormente citados serían una buena solución debido a que son capaces de registrar el movimiento del sensor en sus 6 \ac{DOF}, pero estos dispositivos no están pensados para devolver fuerzas al usuario. Los dispositivos hápticos suelen resolver este tipo de problemas al ser utilizados como dispositivos de seguimiento y a la vez son capaces de reproducir fuerzas e interaccionar con el usuario.

Desde el proyecto \ac{RASimAs} se ha seleccionado el dispositivo \emph{Touch} de la empresa \emph{3D System}\cite{touch} (ver figura \ref{fig:simulator}). Aunque este dispositivo solo sea capaz de proporcionar fuerzas en los 3 \ac{DOF} básicos (X,Y,Z), se ha de alcanzar un compromiso entre versatilidad y el presupuesto. Existen dispositivos hápticos que pueden ofrecer la experiencia de simular la inserción de una aguja, pero su complejidad implicaría un coste inasequible para la comercialización de un simulador. %y aumentaría la dificultad de una inmersión del usuario debido a que no existe dispositivo háptico con esas características de pequeñas dimensiones. Sin embargo, el dispositivo \emph{Touch} proporciona la posibilidad de que el sistema sea más barato y consiga cierto realismo. 
%El usuario maneja este dispositivo a través de una especie de bolígrafo (\emph{stylus}) que es capaz de reconocer la posición, orientación y devolver fuerzas. Incluso es lo suficiente ligero para poder permitir su colocación en otras posiciones con el objetivo de permitir a los usuarios elegir con que mano utilizar los dispositivos.

% Aun así, la utilización de este dispositivo para simular una aguja no esta exento de problemas. El objetivo es simular una aguja real pero el actuador del dispositivo es completamente diferente y sería muy difícil replicar el mismo comportamiento que se experimenta en una situación real.
% Con el objetivo de solventar este problema, se ha intentado adaptar el bolígrafo que dispone para que aloje una aguja real utilizado en el procedimiento.
% El dispositivo \emph{Touch} permite desmontar la punta del actuador con la intención de permitir a otros desarrolladores encajar nuevas creaciones a través de un conector tipo \emph{Jack}. En este caso, desde \ac{SG} se diseñó un modelo 3D que encaje en el conector de manera que la aguja pueda ser agarrada por el usuario. Esto permite que se puedan utilizar el mismo tipo de aguja estandarizando la solución como se puede ver en la figura \ref{fig:needle}.

% \begin{figure}[ht]
%     \centering
%     \includegraphics[width=0.5\textwidth]{IMG/needle.PNG}
%     \caption{Primer prototipo para adjuntar una aguja real al dispositivo háptico.}
%     \label{fig:needle}
% \end{figure}

% Esta solución, aunque bien diseñada, en las primeras pruebas mostró varios factores que hicieron descartarla por parte del comité médico. El desplazamiento entre el centro del conector del dispositivo háptico y el lugar donde el usuario agarraba la aguja junto con la flexibilidad de esta, introducía un desplazamiento virtual que hacía que el simulador no representará fielmente la posición real en el mundo virtual. Desde la \ac{URJC}, se sugirió una adaptación  anteriormente utilizada en un desarrollo anterior\cite{phantompen}. Esta versión encajaba en la forma de bolígrafo que tenía el dispositivo y permitía el uso de una aguja estandarizada de la misma manera como se puede apreciar en la figura \ref{fig:needle2}. Incluso, salvando las distancias, permite al usuario un agarre más realista. Esta versión fue la aceptada para el prototipo a falta de más soluciones y propuestas.


% \begin{figure}[ht]
%     \centering
%     \includegraphics[width=0.5\textwidth]{IMG/needle2.jpg}
%     \caption{Prototipo para permitir sujetar la aguja sin desplazamientos en el dispositivo háptico.}
%     \label{fig:needle2}
% \end{figure}
% \todo{rehacer la imagen de la aguja}



% Otro problema derivado de la selección del dispositivo era que el espacio de trabajo es limitado y cuanto más cerca se trabaja del límite de este espacio, más inestable puede ser el comportamiento del dispositivo. Afortunadamente el procedimiento de \ac{RA} tiene también un espacio limitado de acción y puede simplificarse la solución a este problema. Se propuso crear una mesa de trabajo que limite la acción del usuario al área de interés. 

\subsection{Software}

%\subsubsection{Motor de \ac{RV}}

%Uno de los componentes principales de un simulador de \ac{RV} es el motor donde se producen todos los cálculos que permiten simular la escena virtual. 
%En el contexto de \ac{RASim}, es lógico que este simulador tenga un núcleo que realiza las simulaciones físicas y cálculos necesarios con el objetivo de comunicarse con los dispositivos de entrada y salida. Además, también es necesario un núcleo de simulación que proporcione las imágenes de \ac{US} para conseguir mostrar una simulación realista que pueda ayudar al usuario. Por último, el sistema también deberá \emph{renderizar} la escena con la información de los dispositivos para ayudar al usuario a que alcance mayor inmersión de la escena. 

Debido a que en el proyecto \ac{RASimAs} participan multitud de instituciones, los módulos encargados de cada tarea han sido desarrollados independientemente por cada participante y se presentarán en las siguientes secciones.


\subsubsection{Simulación visual}

\emph{H3D} \cite{sensegraphics2012open} es una librería de código abierto (\ac{GPL})  orientada a la utilización de hápticos en entornos 3D desarrollada por la empresa \emph{Sensegraphics}. 

El módulo \ac{Courseware} utiliza la librería \emph{H3D} como punto de comunicación entre el software y los dispositivos. De esta forma gestionará los demás componentes y dispositivos. A través de este software, también se recopilará la información de los dispositivos de entrada.
En cuanto el simulador es iniciado, \emph{H3D} se encarga de cargar la escena seleccionada utilizando el formato \ac{X3D}, inicia el módulo de simulación física y activa los dispositivos:  el dispositivo háptico y el \emph{\acs{tracker}}. La vista del paciente y la sala de operaciones es \emph{renderizada} por \emph{H3D} a través de su gestor de ventanas.
%Las simulaciones físicas y el cálculo de las deformaciones de los tejidos como consecuencia de la colisión de los instrumentos médicos serán calculadas en el siguiente componte del simulador.

\subsubsection{Simulación física}

En esta ocasión, es el software \emph{SOFA} \cite{sofaweb}, desarrollado por el grupo \ac{INRIA}, el encargado de realizar las simulaciones físicas. Esto permite simular la deformación de los tejidos al contacto de la sonda o la aguja, además de proporcionar la respuesta física a la aguja que será retransmitida al dispositivo háptico.

La librería \emph{SOFA} es código abierto (bajo licencia \ac{LGPL}) y dispone de multitud de métodos y algoritmos matemáticos usados en simulación como distintos \emph{solvers}\footnote{Herramienta matemática para resolver sistemas de ecuaciones lineales.}, modelos físicos como \ac{FEM} y detección de colisiones, simulación de tejidos rígidos o flexibles.

En \ac{RASim}, \emph{SOFA} realiza las siguientes tareas:
\begin{itemize}
    \item Deformación de la piel y otros tejidos producida por la interacción de los instrumentos médicos utilizando el método \ac{FEM} co-rotacional.
    \item La interacción con la aguja a través de diferentes tipos de tejidos, simulando la penetración en la piel y las fascias, además de la fricción de los demás tejidos \cite{needleinsertion}.
   %\item La contracción de los músculos con una respuesta física según sus propiedades mecánicas.
    
\end{itemize}

\subsubsection{Simulación de ultrasonidos} 

En cuanto a la simulación de la imagen de ultrasonidos, se utiliza una librería basada en \ac{ViSTA} desarrollada por el grupo \ac{RWTH}. \ac{ViSTA} es código abierto (bajo licencia \ac{LGPL}) y se utiliza para implementar el algoritmo con el que se simula la generación de imágenes de \ac{US}.

El módulo de ultrasonidos está desarrollado con un algoritmo basado en métodos acústicos geométricos \cite{Law2015}, el cual es clasificado como un enfoque \emph{generativo}. Pensando geométricamente, la onda de sonidos se puede aproximar a través de rayos que interaccionan con elementos de la escena, por consiguiente producen un reflejo del rayo refractado o dispersado. Estos rayos se definen por energía acústica, que puede ser absorbida o disipada durante la interacción con la escena. Esta energía es usada para calcular la intensidad de los ecos y en consecuencia la intensidad de los \emph{píxeles} que servirán para construir la imagen resultante.

Para alcanzar tiempos interactivos en presentar las imágenes de ultrasonidos, la mayor parte de los intensos cálculos y operaciones pesadas son realizadas durante un proceso previo. Específicamente, en esta etapa se genera: un campo acústico que define las propiedades del haz de la sonda y el transductor, las texturas tridimensionales que serán usadas para simular la dispersión del rayo y, además, las estructuras de datos auxiliares que ayudarán a acelerar el proceso de simulación.
De esta manera, cuando el objetivo sea tener una respuesta interactiva, sólo será necesaria la posición y orientación de la sonda virtual que se usará para calcular el origen y la dirección de los rayos acústicos. Estos rayos serán muestreados a través de la escena, estimando la energía reflejada y la distancia recorrida a través de los tejidos. Por otra parte, el remanente de energía transmitida, las absorciones y reflexiones se tienen en cuenta para generar la imagen final.





% \subsection{Arquitectura del simulador}
% \label{rasim:arqrasim}

% Antes de continuar, se va a presentar la arquitectura lógica del simulador para permitir al lector que tenga una visión completa del sistema en su conjunto.

% Siguiendo la arquitectura de un simulador de \ac{RV} tipo que se puede observar en la figura \ref{fig:RVarq}  en el capítulo \ref{art:simulador}, un simulador puede ser definido por una serie de componentes. El usuario utilizará los dispositivos de entrada y salida que se comunicarán con un motor de \ac{RV} que calcula todas simulaciones físicas y visuales según la escena virtual almacenada en la base de datos. El motor actualizará el estado de la simulación según las tareas que hayan sido programadas. A continuación se procede a explicar cada componente por separado. 


%A continuación, se va a proceder a describir aquellos dispositivos que utilizan el canal táctil. \todo{este párrafo va bien aquí?}


\subsubsection{Servidor local y global}

El proyecto \ac{RASimAs} contempla que se almacene toda aquella información generada en la realización del proyecto. En cuanto a la información relevante para el simulador \ac{RASim}, se guardarán las escenas y modelos anatómicos que se utilizarán en el simulador. También, se almacenará el material multimedia complementario que sea necesario para la plataforma de entrenamiento. Además, esta base de datos servirá para guardar el perfil de los usuarios y las métricas que se recogerán para evaluarlos posteriormente.

Con el objetivo de no necesitar estar conectado a Internet permanentemente, se han diseñado un servidor local y otro global. El servidor local será instalado en el simulador de tal forma que periódicamente consulte nuevas actualizaciones o mande información al servidor global que comparten todos los simuladores.

\subsubsection{Courseware}

%Con el objetivo de mostrar la versatilidad y comprobar la hipótesis de esta tesis, se ha contribuido al proyecto \ac{RASimAs} con la generación de un módulo de entrenamiento que permita utilizar los modelos generados por el algoritmo de posicionamiento de pacientes virtuales. El desarrollo de este módulo y su posterior evaluación demostraría la capacidad del algoritmo propuesto en un ejemplo de caso de uso.

%Tanto el motor de \ac{RV} como los dispositivos de entrada y salida no podrían ser capaces de servir como herramienta de aprendizaje por si mismos si no hubiera un software que guía las tareas y las acciones del usuario cuando utiliza el simulador.
El módulo de \ac{Courseware} es una pieza fundamental en \ac{RASim} ya que gestiona la comunicación entre componentes, ayuda y recopila información de los usuarios. Se encarga de recopilar información de los dispositivos y se comunica con la base de datos que almacena las métricas de los usuarios. A la vez, será la aplicación encargada de manejar las ventanas y la interacción del usuario con el sistema. 

El usuario podrá identificarse en el simulador a través de una interfaz. La aplicación recibirá el perfil desde el servidor y basándose en el perfil del usuario, el \ac{Courseware} proporcionará el escenario de entrenamiento según el protocolo que se haya definido. Este módulo, es el encargado incluso de descargar nuevos escenarios del simulador desde el servidor. 

Debido a la complejidad e importancia de este módulo, será descrito en una sección independiente a continuación.


\section{Hypothesis Validation}
\label{conclu:hipotesis}

The proposed algorithm has achieved the objectives we had enumerated in section \ref{intro:objetivos}. A novel technique has been proposed that allows adapting interactively the virtual patient anatomy with internal tissues from their original position to any desired pose. Furthermore, the algorithm will work with incomplete anatomical data and without a mechanical characterisation of the patient tissues. %This is possible even the models are incomplete or a proper mechanical description is not available. %Revisar

In the first use case, the proposed technique has been integrated into the ITGVPH, where the physician supervises the proper animation of the VPH tissues. This use case verifies the flexibility of our algorithm which works without complete models or mechanic properties. In order to prove the usefulness of those VPH in RA training, a courseware for the RASim simulator has been created. Despite the fact that \acs{RASim} has not been formally validated, \acs{RASimAs} medical committee gave their approval. 




%El algoritmo propuesto ha cumplido con los objetivos que se habían enunciado en la sección \ref{intro:objetivos}. Se ha diseñado un cauce de animación esqueletal que permite transformar interactivamente los modelos anatómicos de pacientes virtuales, los cuales originalmente se encuentran en una postura diferente a la posición necesaria para el entrenamiento de un determinado procedimiento médico. A su vez, el cauce permite deformar la anatomía aunque los modelos estén incompletos o falten sus descripciones mecánicas. Aunque no se ha podido realizar una evaluación formal de la \emph{suite} \acs{ITGVPH}, esta si contaba con el visto bueno de los médicos implicados en el proyecto \acs{RASimAs}.
%\todo{explica que contaba con el visto bueno del los medicos implicados en tel proyecoto}
%en el segundo caso de uso si ha sido posible.



Regarding the second use case, %En el segundo caso de uso, si ha sido posible su evaluación formal.
%\todo{Reescribe la parte comentada en otra frase.}
our method allows adapting the pose of the virtual patient and its internal tissues while the system computes its X-ray projection. It is required to be interactive and plausible. Its flexibility makes possible the use of virtual patients from external resources.
%El algoritmo propuesto forma parte del simulador de radiología diagnóstica exige que la adaptación del paciente virtual sea plausible a la vez que se obtiene una imagen radiográfica al instante.
%Este segundo caso de uso representa una exigencia mayor al tener que compartir la deformación producida con otra librería manteniendo las tasas de refresco interactivas.
A combined face and content validation study has been conducted. Generally speaking, face validity shows that our tool is realistic. The overall results of the content validity show both the usefulness and the suitability of our tool in teaching and/or learning X-ray radiography. %La validación de apariencia y de contenido realizada para el simulador %de radiología diagnóstica %permite asegurar que el algoritmo proporciona deformaciones útiles para el entrenamiento de las proyecciones radiológicas.
%Esto induce a pensar que se pueden utilizar enfoques geométricos para el entrenamiento de procedimientos médicos.
Considering the results obtained, we could say that our geometric approach could be integrated into medical training applications and its usefulness has been demonstrated.

In conclusion, taking into account the findings, we could confidently affirm that the starting hypothesis has been validated.
%Considerando esto último, se puede afirmar positivamente que se ha validado la hipótesis de partida.




\section{Contributions}
\subsection{Scientific Contributions}
\label{conclu:cientifica}

\textbf{National Conference paper } \emph{``An Interactive Algorithm for Virtual Patient Positioning"} \cite{ceig.20151197} published in \emph{Congreso Español de Informática Gráfica} in 2015. 

Our first main contribution concerns the first version of the proposed algorithm in a national congress. The objective was to adapt the classic pipeline of skeletal animation to animate the character's internal anatomy. In this publication, the \emph{skinning} technique selected was \acs{DQS} which is the common alternative to \acs{LBS}. Occasionally, \acs{DQS} produces volume gain which is not usefulness in training. We introduced the optimisation stage in order to refine the final result sacrificing interactive rates.  It can be found in the appendix \ref{anexo:ceig}.

%La primera contribución fue presentar una versión inicial del algoritmo en un congreso nacional. El objetivo era la adaptación del cauce clásico de animación esqueletal a modelos volumétricos. En esta publicación, la técnica de \emph{skinning} utilizada es \acs{DQS} que, aunque es una alternativa popular a la técnica \acs{LBS}, generaba incrementos de volumen que no eran útiles para utilizarlos en entrenamiento médico. De esta manera se introdujo un proceso de optimización no interactivo a cambio de mejorar la calidad del resultado. Mientras se trabajaba en las distintas integraciones del algoritmo en los casos de uso presentado, se buscó alternativas a la técnica \acs{DQS}.

\textbf{Journal paper }\emph{``Real-time animation of human characters' anatomy"}\cite{SUJAR2018268} published in \emph{Computer \& Graphics} in 2018. 

%\todo{Paper EG y el paper pendiente de publicacion.}
As the main contribution of this thesis, we presented the proposed algorithm in a peer review journal. As it can see in results (sec. \ref{posing:result}), we have selected \acs{COR} \emph{skinning} technique in order to replace the optimisation stage. It can be read in the appendix \ref{anexo:cag}.
%Como contribución principal del esta tesis, se presentó un artículo en una revista impactada en el que se presentaba el algoritmo propuesto. Como se ha descrito en los resultados del capítulo (sec. \ref{posing:result}), la solución propuesta al utilizar \acs{COR} podría reemplazar la fase de optimización con la consecución de que la herramienta permita deformaciones útiles para el entrenamiento médico interactivamente.

\textbf{International Conference paper }\emph{``A Virtual Physiological Human Model for Regional Anaesthesia"}\cite{VHZKLBSGSD16} published in \emph{Virtual Physiological Human (VPH)} in 2016.

In this conference paper, we have presented the objectives of the \acs{ITGVPH}, where the \acs{TPTVPH} was integrated and described its functionality. This article would be the beginning of a series of papers, but the absence of formal validation of the toolkit and the unexpected problems in \acs{RASim} prevent to publish more in this research line.

%Este artículo representa la propuesta de los objetivos de la \emph{suite} \acs{ITGVPH} o dentro del proyecto \acs{RASimAs}. La idea principal era crear un conjunto de herramientas que permita crear una base de datos que contenga multitud de variaciones anatómicas. En este conjunto se ha integrado el algoritmo propuesto a través de la herramienta \acs{TPTVPH} que permitía la adaptación del paciente virtual a la postura requerida por el simulador. Se publicó inicialmente esta publicación en el congreso \emph{ Virtual Physiological Human (VPH)} como introducción a las demás publicaciones relacionadas con el proyecto \acs{RASimAs}. Lamentablemente, debido a los problemas para realizar la validación del simulador junto con la finalización del proyecto, han  impedido seguir en esta línea. %Tampoco ha sido posible publicar  hasta que el simulador haya pasado su validación clínica.

\textbf{International Conference paper }\emph{``gVirtualXRay: Virtual X-Ray Imaging Library on GPU"}\cite{sujar:hal} published in \emph{Computer Graphics and Visual Computing} in 2017.

This contribution is a collaboration with \emph{Franck P. Vidal}, who leads the \emph{gVirtualXRay} project \cite{gVirtualXRay}. This paper describes the potential of the  \emph{gVirtualXRay} framework and introduces the initial proposal of X-ray simulator. It can be found in the appendix \ref{anexo:cguk}. It was extended with a new manuscript called: %Esta contribución surge como colaboración con \emph{Franck P. Vidal} \cite{gVirtualXRay}, el cual dirige un proyecto de simulación de rayos X \emph{gVirtualXRay}. Esta colaboración dio lugar al segundo caso de uso de esta tesis. En este artículo, se presentó las posibilidades de la librería de \emph{gVirtualXRay} para ser utilizada como simulador de radiología que será finalmente completada con la publicación del artículo en progreso llamado: 
\emph{``Interactive learning environment for diagnostic radiographywith real-time X-ray simulation and patient positioning"}, already submitted to \emph{Computerized Medical Imaging and Graphics} journal. The manuscript can be found in the appendix \ref{anexo:xray}.

%con intención de ser publicado en la revista impactada Computerized Medical Imaging and Graphics en un futuro cercano. Se puede consultar un manuscrito en el anexo \ref{anexo:xray}.



\subsection{Technical Contributions}
\label{conclu:tecnica}

%Our technical contributions can be summarised as follow:

%Las contribuciones tecnicas son los simuladores y las herramientas. 1. Herramienta de poses. Integración en el sistema de generación de pacientes virtuales. 2. cOURWARE DE RASIMAS. 3. Colaboraste en la integración de RASim. Propuesta de prototipo sin haptico que si conto con un opinon positiva de los medicos (!!!). Di que reescribiste los drivers del haptico

%4. Simulador de rayos x: Sistema de pose, integracion con el simulador del rayos x, courseware.

%\todo{Aqui no hables de paper. Habla de herramientas desarrolladas, librerias: Libreria de posing, integradada con GMRVGL (libreria de render desarrollada en el grupo), herramienta de posing. Integracion de la herramienta de creacion de pacientes virtuales. Integración de RASim. Adapatacion del prototipo a Flockof birds (has reescrito los drivers). Herramienta de radiologia diagnostica.}

\textbf{System:} Animation framework of virtual patient.

The main technical contribution is the implementation of the proposed animation pipeline based on the framework \emph{GMRVGL} developed by the research group \acs{GMRV} at \acs{URJC}. A UI has been developed to allow users to adapt the pose.
%A partir de la librería de \emph{render} \emph{GMRVGL} desarrollada por el grupo de investigación \acs{GMRV} de la \acs{URJC}, se desarrolló el cauce del algoritmo propuesto. Se desarrolló una \acs{IU} que permite al usuario adaptar la posición de manera sencilla, dando lugar también a la herramienta \acs{TPTVPH} que se incorporó en la \emph{suite} \acs{ITGVPH}.


\textbf{System:} \acs{RASim}'s Courseware

An application has been developed to provide trainees and physicians with supportive information for self-directed learning and training. This system guides users through the procedure, while it manages all components of the simulator and retrieves metrics about users performance. With that, it provides formative and summative feedback.
%Se ha desarrollado una aplicación de autoevaluación y aprendizaje del procedimiento de \acs{RA}. Esta aplicación dirige al usuario a través del procedimiento, mientras coordina todos los módulos del simulador en función del tipo de entrenamiento. Debido a esto, se ha colaborado activamente en la integración del prototipo \acs{RASim}.


\textbf{System:} \emph{Flock of birds}' Driver

In order to provide an alternative to the faulty haptic devices, we proposed to use magnetic trackers. At \acs{URJC}, the only device available was an old version of \emph{Flock of birds} which is not supported by moderns operative systems. The device's driver was re-implemented to communicate with \emph{H3D} and the virtual scene.

%Con el objetivo de proponer una solución alternativa a los dispositivos hápticos defectuosos, se propuso la utilización de \acs{tracker}s magnéticos. El único dispositivo disponible en la \acs{URJC} era el \emph{Flock of birds}, el cual no contaba con soporte para sistemas operativos modernos. Se reimplementó los controladores de este dispositivo para poder comunicarse con el módulo \emph{H3D} encargado de la comunicación de los dispositivos y su representación en la escena virtual.


\textbf{System:} X-ray projectional radiography

Bringing together the animation algorithm and the \emph{gVirtualXRay} framework, an X-ray simulator have been developed which provides to students and teachers an interactive learning environment for diagnostic radiography. We have worked together with Dr. Franck P. Vidal to introduce modifications in the \emph{gVirtualXRay} in order to share \acs{GPU}'s memory.

%Junto con la librería de posicionamiento de paciente virtuales y la librería \emph{gVirtualXRay} se ha desarrollado un simulador de radiología diagnóstica que permite a estudiantes y profesores conseguir la radiografía de un paciente virtual mientras seleccionan la postura adecuada. Se trabajo conjuntamente con el Dr. Franck P. Vidal para incorporar las modificaciones necesarias en \emph{gVirtualXRay} para compartir la memoria en \acs{GPU}. 





%\todo{Mete como contribucion cientifica el ultimo paper, ponlo como apendice y di que está pendiente de publicar. }


\section{Limitations and Future Work}
\label{conclu:future}

\subsection{Algorithm for Virtual Patient Positioning}

Our technique follows a geometrically-based approach.  Therefore, it provides a heuristic solution. For this reason, this technique is not suitable for surgical planning or applications which need a realistic behaviour. For example, our algorithm does not take into account the effect of gravity and produces the same deformation no matter if the patient is standing up or lying. Alternatively, medical trainers do not need a specific real patient model but a set of anatomically different patients. Our system provides plausible poses for training and educational purposes.

%La principal limitación que presenta el algoritmo de posicionamiento de pacientes virtuales es la utilización de un enfoque geométrico. Como tal, los resultados que proporcionan son heurísticos y no físicamente correctos. Esta característica impide que pueda emplearse para planificación quirúrgica o en aquellos lugares que necesiten un comportamiento fiel a la realidad. Por ejemplo, no se tienen en cuenta efectos como la gravedad en los tejidos internos, lo cual generaría transformaciones en los tejidos anatómicos diferentes si el paciente virtual se encontrara de pie o tumbado.

Another limitation is the requirement of minimum structures. The algorithm needs skin and bones correctly labelled. Those are the most usual tissues registered by medical imaging. Skin and bones will be used to control the pre-process stage.
%Otra limitación es la necesidad de que existan unas estructuras mínimas. El algoritmo requiere por diseño que los tejidos de la piel y los huesos estén correctamente identificados. Aun así, estos tejidos son los más habituales que se pueden capturar en muchas de las imágenes médicas, por tanto, servirán para guiar el proceso e identificar algunos puntos claves para el correcto funcionamiento del algoritmo.
    
Due to tissue extraction of medical imaging are not perfect, anatomic models usually have self-collisions and collisions between tissues. Our algorithm is robust and it can manage them, but it does not solve them. Despite the fact that tissues outside the skin are unreal, the algorithm deals with them like the rest of tissues.
%Debido a que la extracción de los tejidos desde imágenes médicas no es perfecta, una situación crítica son las auto colisiones y colisiones entre tejidos. Aunque el algoritmo es robusto al tratamiento de estas, no está diseñado para resolver colisiones. Aunque aquellos tejidos que traspasen la piel son irreales, se tratarán como el resto de tejidos sin poder asegurar una deformación realista.
    
%El algoritmo utilizar un esqueleto virtual para definir los movimientos de las articulaciones. Se utilizará la información que proporciona los tejidos óseos para construir un esqueleto virtual adecuado al modelo de entrada. La principal limitación es la selección manual de las zonas identificadas que se necesitan para identificar los puntos de rotación de la articulación y su sistema de referencia.


%Aún así, los resultados obtenidos son lo suficientemente plausibles para proporcionar un método que consiga generar modelos anatómicos en el contexto de entrenamiento médico.

It is worth to mention that the proposed pipeline allows updating all techniques in each stage independently. In the future, steps can be replaced by new algorithm in order to improve the results.
%Se puede destacar que el diseño seguido del cauce de animación esqueletal permite actualizar los algoritmos utilizados en cada etapa independientemente. En el futuro pueden sustituirse etapas por nuevos métodos que permitan mejorar los resultados. 
Currently, the first stage (\emph{rigging}) is a specific solution where the interesting zones, which are used to calculate the joint's reference system, are manually selected once in the reference model (before adapting the specific virtual patient data). A technique can be developed to extract automatically a virtual skeleton from the bone's anatomic models, it may be base on \cite{Tagliasacchi}.
%Por ejemplo, en la etapa de \emph{rigging}, la principal limitación es la selección manual de las zonas identificadas que se necesitan para calcular los puntos de rotación de la articulación y su sistema de referencia. Se podría desarrollar un método que extrajera el esqueleto virtual automáticamente basándose en el tejido óseo del paciente virtual, inspirado en el trabajo de \cite{Tagliasacchi}.
In relation to the weighting step, more complex techniques like \cite{Jacobson:2011} could be evaluated if they produce better results.
%En cuanto la etapa de pesado, se podría analizar si la incorporación de técnicas de pesado más complejas como \cite{Jacobson:2011} ofrecerían mejores resultados.
New \emph{skinning} techniques could produce better deformations which solve collisions and self-collisions like \cite{Vaillant:2014} as a starting point.
%Un objetivo para mejorar la calidad de los modelos anatómicos generados es la búsqueda de nuevas técnicas de \emph{skinning}. Podría buscarse métodos que pudieran solventar los problemas de las auto colisiones como \cite{Vaillant:2014}.
On the selection pose, inverse kinematic \cite{Shi:2007} can be added seamlessly into the system. Movements more complex than rotations of the joints, like splines, could be another alternative to integrate \cite{joints}.
%En la selección de poses, la técnica de cinemática inversa \cite{Shi:2007} se podría añadir de manera trivial para mejorar la interacción del usuario con el sistema. Otra alternativa sería incorporar movimientos de articulaciones más complejos \cite{joints} que las simples rotaciones.
Even so, we could replace it with physically-based methods like \emph{Point-Based Dynamics} \cite{abu2015position} if they will be able to animate in real time or define stiffness for each tetrahedron.
%Incluso, se podría incorporar nuevos métodos basados en físicas como los prometedores métodos \emph{Point-Based Dynamics}  \cite{abu2015position} que permitan animación en tiempos interactivos y la caracterización mecánica de los tetraedros.

\subsection{RASimAs}

Although every module of the simulator has been positively evaluated by medical partners, there was no clinical trial in order to perform a construct validation study to assess the actual benefit of using this simulator.
%Gracias a que los distintos módulos han sido evaluados por parte de los médicos y el prototipo ha obtenido una satisfactoria validez de contenido, el proyecto \acs{RASimAs} fue calificado de manera favorable. Aun así, la finalización del proyecto y la falta de fondos han hecho imposible la realización de más validaciones de constructo, concurrente y predictiva en un entorno clínico.    %Se espera poder solucionar los defectos que presentan los dispositivos o incorporar otros diferentes con el fin de poder llevar el simulador a los entornos clínicos.

The main limitation is that \acs{RASim} have got only one example of a nerve block. Despite the fact that femoral nerve block is one of the most clinically simplest procedure, one goal of the project is to train with a variety of nerve block. The finalisation of the \acs{ITGVPH} would amend the lack of anatomic models in the simulator.
%La principal limitación del prototipo \acs{RASim} es que dispone solamente de un caso de bloqueo de nervio. Si bien el caso del bloqueo femoral, es el más común de los practicados por los anestesistas, una de las previsiones del proyecto era la posibilidad de entrenar varios tipos de bloqueos. La finalización  de la \emph{suite} \acs{ITGVPH} debería suplir la falta de modelos para utilizar en otros bloqueos.

Furthermore, all software modules running at the same time combined with their complexity make a high computational cost. As a result, volumetric meshes are coarse and they have got low resolution. This avoids incorporating physics effects such as hidrodisection or diffusion analgesic. A proposed solution is to improve the integration between software modules and its communications.


%Por otra parte, la suma de los módulos software que se están ejecutando al mismo tiempo junto con la propia complejidad de estos, hacen que el coste computacional sea muy alto. La consecuencia es la utilización de mallas groseras y con poco detalle. Esta limitación ha impedido que se incorporaran efectos físicos como la hidrosección o difusión del analgésico. Como posible solución, se podría mejorar la comunicación e integración entre los módulos software.

Touch haptic device has got a limited workspace which constrains the user's movements. Although there are better devices with a large range, their higher cost makes Touch device the best option after solving their manufacturing defects.
%El propio dispositivo háptico presenta un limitado espacio de trabajo que restringe los movimientos de los usuarios. Aunque existen dispositivos con un rango de alcance más amplio, su  alto coste hace que el dispositivo actual sea la mejor opción una vez resueltos sus defectos de fábrica.

%Además, este entorno se podría adaptar a otro tipo de procedimientos médicos. 

In relation to the \acs{Courseware} module, its main limitation is to be specifically designed to \acs{RA} procedure. Even so, load functionality of external resources and help widgets could be recycled to use them in another procedure. Additionally, new metrics should be defined specifically to the new training.

%En cuanto al módulo \acs{Courseware}, su limitación recae en su desarrollo específico para el procedimiento de \acs{RA}.   %Aun así, se diseñó para cargar el material multimedia externamente, por lo cual, además se podría reutilizar aquellas funcionalidades para que solo fuera necesario generar nuevo contenido. 
%Aun así, podría reciclarse todo lo relacionado a cargar el material multimedia externamente, y también sería posible adaptar las ayudas de usuario ya desarrolladas para otro tipo de procedimientos. Finalmente, esto implicaría del mismo modo la necesidad de volver a definir nuevas métricas interesantes específicamente para el nuevo entrenamiento.

Finally, it is worth to mention \acs{RASim} does not take protocol aspects into consideration. Room or medical equipment set up is important to the procedure. Users may interact with the virtual scene. Other problems are the aseptic technique or gel application on the \acs{US} probe. As a first option, a \emph{paint} functionality could be implemented in the virtual patient model using the same simulator's devices.
%Por último hay que destacar que en \acs{RASim} no se tienen en cuenta algunos procedimientos protocolarios. La preparación de la sala o el instrumental médico son aspectos importantes del procedimiento, lo cual se podría incorporar mediante interacción del usuario con el escenario virtual. Otro aspecto fundamental del procedimiento es la técnica aséptica o la aplicación del gel para la sonda de \acs{US}. Como primera opción, podría implementarse un efecto de pintado en la piel del paciente virtual utilizando los propios dispositivos del simulador.


\subsection{X-ray projectional radiograph simulator}

%Aunque la aplicación se ha diseñado para satisfacer las necesidades de los radiólogos, presenta ciertas limitaciones respecto al procedimiento de diagnóstico por imagen. Existen cuestiones relacionadas con la preparación del paciente para el procedimiento que no pueden ser cubiertas por el simulador, que se pueden ver a continuación:

This simulator has been designed to satisfy radiographer needs. Nevertheless, this tool has some limitations about the full procedure. There are important concerns like patient preparation which is not covered in this tool.
\begin{itemize}
 \item Physicians must employ appropriate and effective communication with patients.

 \item Patients are indicated about remove clothes or artefacts over the relevant examination area.

 \item Recommendations about use lead rubber either on patients or radiographers.

 \item Some assessments which involve medical conditions or protocols like pregnancy, correct patient identification, etc.
 
 \end{itemize}

% \begin{itemize}
%     \item Se recomienda utilizar un lenguaje apropiado y efectivo en la comunicación con los pacientes que no es posible recoger por el simulador.
%     \item No se representa la necesidad de indicar al paciente acerca de quitarse prendas u objetos que afecten al área de examen.
%     \item No se practican las recomendaciones en cuanto al uso de protectores de plomo para pacientes y profesionales.
%     \item Otras comprobaciones que implican condiciones médicas o protocolarias como embarazo, identificación del paciente, etcétera.
% \end{itemize}
In order to solve some of the previous protocol aspects, a set of questions could be created asking for the medical history of the virtual patient, or a voice recogniser to train the appropriate language.
%Podría valorarse incluir una serie de preguntas a cerca de un informe médico creado para el paciente virtual, o incluir un reconocimiento de voz para practicar el lenguaje utilizado.


Our X-ray simulator shares the same limitations with the positioning algorithm. Improvements in the proposed technique will benefit the X-Ray simulator. The positioning system sacrifices accuracy in favour of computational performance and flexibility. For example, natural phenomena such as gravity cannot be simulated. As a result of using a geometrically-based algorithm, its plausible positions are only oriented for training and educational purposes.


%Además, el simulador cuenta con las limitaciones propias de los módulos por los que esta compuesto. Todas las mejoras introducidas en el algoritmo propuesto que tengan relación con la calidad de las transformaciones, serán beneficiosas para conseguir imágenes anatómicamente correctas. Por ejemplo, el tratamiento de las autocolisiones o la diferencia de los tejidos según el paciente virtual se encuentre de pie o en décubito.

On the other hand, the simulator shares the limitations of \emph{gVirtualXRay} as well. Due to the deterministic nature of the X-ray simulation algorithm, mAs cannot be taken into consideration. It is possible to replicate a low mAs value by adding \emph{Poisson} noise to the X-ray image, however this parameter is quite important in the clinical environment to improve radiography quality.

%Por otra parte, se encuentran las propias limitaciones de la librería \emph{gVirtualXRay}. La configuración de la intensidad \acs{mAs} de la máquina de rayos X no está incorporada en el sistema, y es un parámetro importante para los radiólogos con el fin de mejorar la calidad de la radiografía. En un principio, se ha introducido un ruido de \emph{Poisson}, pero este método no es capaz de simular todos los valores de intensidad utilizados en la realidad. La característica determinista de la librería no hace posible una solución sencilla.

In order to increase the number of available virtual patients,  \emph{gVirtualXRay} could be updated to load volumetric representations, as positioning algorithm allows. 

%Para aumentar la cantidad de pacientes virtuales disponibles para la herramienta, se podría actualizar la librería para incorporar la funcionalidad de producir radiografías con modelos volumétricos, representaciones que el algoritmo propuesto si puede manejar.

Another extra feature is to measure the radiation doses received by the patient. Currently, there are online calculator doses  \cite{xraydose} which could be integrated into the simulator.
%Otra funcionalidad extra que se podría incluir en el simulador, es la medición de la dosis recibida por el paciente. Existen actualmente simuladores de dosis y calculadoras \emph{online} \cite{xraydose}, que podrían ser incorporadas al simulador.

In the same way, other medical imaging processes can be simulated and integrated into the same tool. Recently, there are projects trying to simulate effectively techniques like MRI (magnetic resonance imaging) or CT (computed tomography), which requires some patient position knowledge. 

%Incluso, sería posible incorporar otras técnicas de imagen médica que puedan ser simuladas. Se podría ampliar el abanico de tecnologías de diagnóstico por imagen (p. ej. \acs{IRM}, \acs{US} o \acs{TC}) que requieran posicionar al paciente virtual obteniendo una imagen médica al instante, lo cual podría derivar en una nueva línea de investigación.